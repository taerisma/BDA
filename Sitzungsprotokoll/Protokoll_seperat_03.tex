\documentclass[10pt,a4paper]{article}
\usepackage[utf8]{inputenc}
\usepackage[german]{babel}
\usepackage[T1]{fontenc}
\usepackage{amsmath}
\usepackage{amsfonts}
\usepackage{amssymb}
\usepackage{graphicx}
\usepackage[left=2cm,right=2cm,top=2cm,bottom=2cm]{geometry}

%\documentclass[10pt,a4paper]{article}
%\usepackage{layout}
\begin{document}
%\chapter{Gesprächsprotokoll}
%\maketitle
\section{SW3}

\subsection{Gesprächsthemen}

%Version:1 \\
Es soll bereits jetz mit einer ersten Simulation einer Dipol Antenne für das 2.4GHz ISM Band begonnen werden. Die Impedanz der beiden symetrischen Antennen soll betrachtet werden. Das Verhalten der Antennen soll untersucht werden wenn die mechanische Länge von $\lambda/2$ auf $\lambda /10 $ abnimmt. \\
Die Theorie des $R_{rad}$ soll mit den Simulationen verglichen werden.\\


Für die BDA wichtige Punkte die aus dem Gespräch hervorgegangen sind:
\begin{itemize}
	\item Simulation einer Dipol Antenne
	\item Simulation einer Loop Antenne
	\item Das verhalten von $R_{rad}$

\end{itemize}

\subsection{Wichtige Beschlüsse}

\begin{itemize}
	\item Start mit Empire Xccel einfache symetrische Antennen simulieren
	\item Email an Herr Lerch betreffend des WiFi Moduls
	\item Bis Ende nächster Woche die Einleitenden Kapitel der Doku fertig stellen
	\item Für die BDA relevanten Theoriebereich dokumentiert
	\item IEEE für Frequenzbereich des Low Energie Bluetooth
\end{itemize}

\subsection{Fragen auf SW3}
\end{document}