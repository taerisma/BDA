\documentclass[10pt,a4paper]{article}
\usepackage[utf8]{inputenc}
\usepackage[german]{babel}
\usepackage[T1]{fontenc}
\usepackage{amsmath}
\usepackage{amsfonts}
\usepackage{amssymb}
\usepackage{graphicx}
\usepackage[left=2cm,right=2cm,top=2cm,bottom=2cm]{geometry}

%\documentclass[10pt,a4paper]{article}
%\usepackage{layout}
\begin{document}
%\chapter{Gesprächsprotokoll}
%\maketitle
\section*{SW8}

\subsection*{Gesprächsthemen}

%Version:1 \\
Kurze Besprechung vor der Zwischenpräsentation.\\
Die bereits erstellen Folien werden besprochen. Die der Inhalt der Präsentation wird verdichtet.\\
Mehr Fokus auf das visualisieren der Antennenausrichtung und die damit zusammenhängenden elektromagnetischen Felder.\\
Der \glqq ist und soll Vergleich\grqq  der Präsentation soll den bereits erarbeiteten Stand klar aufzeigen. Die Präsentation und die daraus folgenden Gespräche mit Experten und Industriepartner soll die noch zu erarbeitenden Aufgaben definieren.\\
Im Gespräch soll ich klar aufzeigen können welche Punkte im Rahmen der BDA noch erreicht werden und welche nicht.


\vspace{10 mm}
Für die BDA wichtige Punkte die aus dem Gespräch hervorgegangen sind:
\begin{itemize}
	\item Aufzeigen welche Arbeiten   in der Verbleibenden Zeit noch gemacht werden
\end{itemize}

\subsection*{Wichtige Beschlüsse}
Inhalt des Antennen Grundladen Teils
\begin{itemize}
	\item Präsentationmit mehr Grafiken einbinden
	\item Präsentation mehr Simulationsergebnisse einbinden
	\item Experten und Industriepartner an die Zwischenpräsentation erinnern
\end{itemize}
\subsection*{Fragen}
\begin{itemize}
	\item Empire Xccel ein Element zum Beispiel eine Spule parallel zur Quelle einbinden
\end{itemize}
\end{document}
