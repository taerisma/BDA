\documentclass[10pt,a4paper]{article}
\usepackage[utf8]{inputenc}
\usepackage[german]{babel}
\usepackage[T1]{fontenc}
\usepackage{amsmath}
\usepackage{amsfonts}
\usepackage{amssymb}
\usepackage{graphicx}
\usepackage[left=2cm,right=2cm,top=2cm,bottom=2cm]{geometry}

%\documentclass[10pt,a4paper]{article}
%\usepackage{layout}
\begin{document}
%\chapter{Gesprächsprotokoll}
%\maketitle
\section*{SW13}


\subsection*{Kein Grspräch}

%Version:1 \\
\textbf{KW 49}


Die verschiedenen simulierten Sipolantennen wurden ausgemmessen . Die Impedanz der Antennen wurde im Freiraum und verbaut in dem "Connect 1" mit dem Netzveranalysator aufgezeichet. Der Wirkungsgrad und die 3D Abstrahlcharakteristik wurde im dem StarLab aufgezeichet.\\
Die S11 Charakteristik der verschieden Antennen wurde aus dem Empire XPU exportiert und mit den S11 Parmaeter des Netzveranalysator in einer gemeinsamen Grafik aufgezeit.

\vspace{10 mm}
Für die BDA wichtige Punkte sind:
\begin{itemize}
	\item qwd
	\item wef
	
\end{itemize}

\subsection*{Wichtige Beschlüsse}
sadcfd
\begin{itemize}
	\item sed
\end{itemize}
\subsection*{Fragen}
\begin{itemize}
	\item ...
\end{itemize}
\end{document}


