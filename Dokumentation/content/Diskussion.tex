\chapter{Diskussion}
Vergleich aus der Theorie\\
den Simulationen der Deigns\\
effektive Resultate\\

Daraus die Folgerung\\

Rückblick auf die konkreten Vorderung der Aufgabenstellung

Im Kapitel \ref{sec:Implementierung} wird gezeigt, wie das Funktionsmuster erstellt wurde, sowie wie die Dipol Antenne  im Fluginstrument Positionierung ist. Im vorhergehenden Kapitel \ref{sec:Messung} wurde erleutert, welche Messsystem verwendet wurden um die Antennenparamter aufzunehem. Ebenfall wurden die erziehlten Messresultate aufgeführt.\\
In diesem Kapitel soll nun die erhalten Resultat mit den Erwartungen aus der Simulationen und der Theorei aufgezeigt werden.

Aus dem Kapitel \ref{sec:Interpretation_Dipol} ist bekannt, das als Funktionsmuster eine Dipolantenne mit der Länge L = 50.25 mm und der Breite 3 mm gerfertigt wird. 

Die Abbildung \ref{S11_Messung_Simulation_Dipolantenn_Freiraum} im Kapitel Messergebinsse zeigt den erwarteten $S_{11}$ Verlauf aus den Simulationen der 3 mm breiten und 50.25 mm langen Dipol Antenne.  Zu erkenn ist, dass die Simulation des Verlaufs im Gerät als bleue Kurve dargestellt, eine erhblich kleinere Rückflussdämpfung, als die rote Kurve der Simulation des Dipols im Freiraum, aufweisst. Eine -10 dB-Bandbreite ist nicht auszumachen, da der $S_{11}$ Wert den nie einen Wert grösser als |-5.25| dB bei 2.44 GHz aufweisst. Die effektive Messung des des $S_{11}$ Wertes ist als hellgrüne Kurve dargestellt. Es ist ersichtlich, das die resonazfrequenz des im Gerät imulierten Dipols sehr gut mit der realität übereinstimmt. Das bedeutet mit dem bestehenden Modell des "Connect 1" Geräts in der Simulationssoftware sehr gut geeignet ist um, die Verschiebung der Resonanzfrequenz durch die materialspeziefischen Eigenschaften zu machen. Ein ganz anderes Bild zwigt sich bei dei den Amplituden der Rückflussdämpfung. Währende die Resnonazfrequenz der Simulation und der Messung sehr gut übereinstimmen, sind die gemessen |$S_{11}$|-Werte viel grösser als jehne der Simulation. Eine erklährung weshalb die $S_{11}$ so überraschenden gut sind liegt wahrscheinlich an der Antennenimpedanz $Z_{ant}$. Diese Zeit im Smith-Diagramm eine Impedanz von (30+j4) $\Omega$. Daher kommt eine gute Anpassung an die Quellenimpedanz des Netzwerkanalysators mit einer Ausgangsimpedanz von (50+j0) $\Omega$ zusatand. Die hellgrüne Kurve zwigt weiter eine -10 dB-Bandbreite von 190 MHz und bei der die Resonazfrequenz $f_{res}$ liegt bei 2.44GHz, also im Zentrum des Bluetooth Antennenspekrums. Die Simulierte Antennenimpedanz $Z_{ant}$ zeigte eine maximale Rückflussdämpfung von   


Aus der Theorie erwartet man eine torusförmige Abstrahlung radial um die Antenne. Weiter ist aus der Theorie bekann, dass das E-Feld der Dipolantenne viel Energie im reaktiven Nahfeldspeichert, wenn Kunststoff sich im Nahffeld befindet. Die Simulationen zeigen ebenfalls, dass die Feldausbreiung nicht kreisrund ist, daher ist die Abblidung \ref{fig:Schnittgemessen} in Kapitel {sec:Messergebinsse} nicht verwunderlich. Sie zeigt Den Gewinn der Antenne in der xz-Ebene.\\
Aus der Abbildung \ref{fig:DipolEFerd} ist das E-Feld einer Dipol Antenne gezeigt. Je eine Nullstelle des elektromagnetischen Feldes ergeben sich in der Verlängerung der Dipolachse. Eine eindeutig Nullstelle lässt sich aus der Abbildung \ref{fig:Schnittgemessen} nicht lesen. Jedoch auffällig ist, dass genau entlang der positiven nudn negativen y-Achsen ausrichtung eine Einbuchtung der Feldstärkte ersichtlich wird. Eine erklährung für die starkte Siganldäpfung entlang der positven und negativen x-Achse in der xy-Ebene könnten wider bei den absorbierenden Eigenschaften des Geräts sein. Im Zentrum des $\lambda/2$-Dipols hescht immer die kleinsete Spanunnung. Ist der Dipol in Resonaz so titt an der Speisestelle ein Kontenpunkt auf, das könnte ein Erklährung für die schwache Feldabstrahlung im Uhrsprung sein.