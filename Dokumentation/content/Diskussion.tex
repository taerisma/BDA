\chapter{Diskussion}
Im Kapitel \ref{sec:Implementierung} wird gezeigt, wie das Funktionsmuster erstellt wurde, sowie wie die Dipol Antenne  im Fluginstrument Positionierung ist. Im vorhergehenden Kapitel \ref{sec:Messung} wurde erläutert, welche Messsystem verwendet wurden um die Antennenparamter aufzunehmen. Ebenfalls wurden die erzielten Messresultate aufgeführt.\\
In diesem Kapitel soll nun die erhalten Resultat mit den Erwartungen aus der Simulationen und der Theorie aufgezeigt werden.

Aus dem Kapitel \ref{sec:Interpretation_Dipol} ist bekannt, das als Funktionsmuster eine Dipolantenne mit der Länge L = 50.25 mm und der Breite 3 mm gefertigt wird. 

Die Abbildung \ref{S11_Messung_Simulation_Dipolantenn_Freiraum} im Kapitel Messergebnisse zeigt den erwarteten $S_{11}$ Verlauf aus den Simulationen der 3 mm breiten und 50.25 mm langen Dipol Antenne.  Zu erkenn ist, dass die Simulation des Verlaufs im Gerät als blaue Kurve dargestellt, eine erheblich kleinere Rückflussdämpfung, als die rote Kurve der Simulation des Dipols im Freiraum, aufweist. Eine -10 dB-Bandbreite ist nicht auszumachen, da der $S_{11}$ Wert den nie einen Wert grösser als |-5.25| dB bei 2.44 GHz aufweist. Die effektive Messung des $S_{11}$ Wertes ist als hellgrüne Kurve dargestellt. Es ist ersichtlich, das die Resonanzfrequenz des im Gerät simulierten Dipols sehr gut mit der Realität übereinstimmt. Das bedeutet mit dem bestehenden Modell des "Connect 1" Geräts in der Simulationssoftware sehr gut geeignet ist um, die Verschiebung der Resonanzfrequenz durch die materialspezifischen Eigenschaften zu machen. Ein ganz anderes Bild zweigt sich bei den Amplituden der Rückflussdämpfung. Währende die Resonanzfrequenz der Simulation und der Messung sehr gut übereinstimmen, sind die gemessen |$S_{11}$|-Werte viel grösser als jene der Simulation. Eine Erklärung weshalb die $S_{11}$ so überraschenden gut sind liegt wahrscheinlich an der Antennenimpedanz $Z_{ant}$. Diese Zeit im Smith-Diagramm eine Impedanz von (30+j4) $\Omega$. Daher kommt eine gute Anpassung an die Quellenimpedanz des Netzwerkanalysators mit einer Ausgangsimpedanz von (50+j0) $\Omega$ zustande. Die hellgrüne Kurve zeigt weiter eine -10 dB-Bandbreite von 190 MHz und bei der die Resonanzfrequenz $f_{res}$ liegt bei 2.44GHz, also im Zentrum des Bluetooth Antenne Spektrums. Die Erwartung aus der Simulation des Dipols im Gerät war eine maximale Rückflussdämpfung von -5.25 dB bei 2.44 GHz zu entnehmen. Die Simulation zeigte weiter, eine Antennenimpedanz $Z_{ant}$ von (16+j14.7) $\Omega$ dies führte zu einer Simulierten Abstrahleffizienz von $\eta$ = 65 $\%$.\\
Die Abstrahleffizienz $\eta$ konnte mit dem StarLab gemessen werden. In der Abbildung \ref{S11_Messung_Simulation_Dipolantenn_Freiraum} im Kapitel \ref{sec:Messergebinsse}  ist die gemessene Abstrahleffizienz überein einen Frequenzbereich von 2.4 bis 2.5 GHz gezeigt. Bei der Zielfrequenz von 2.45 GHZ ist eine Effizienz von 49 $\%$ abzulesen.\\
Die gemessene Abstrahleffizienz kommt noch nicht an die Simulierten Werte heran. Jedoch zeigt sich, dass ohne speziellen Aufwand in Form eines Anpassnetzwerks eine wesentlich höhere Abstrahleffizienz mit einem geeigneten Antennendesign erreicht werden kann, als mit den bis anhin verwendeten Monopol Antenne.\\
Dieser um die Abstrahleffizienz weiter zu optimieren kann das Antennendesign weiter optimiert werden, dass die Antennenimpedanz $Z_{ant}$ der komplexen Impedanz des Transceivers (70+j30) $\Omega$ entspricht. Um Wellenanpassung zu erreichen wird demnach eine Antenne mit 70 $\Omega$ Realanteil und einem induktive Imaginäranteil von 30 $\Omega$ gefunden werden. Eine weitere Möglichkeit bildet die Phasentransformation mit Hilfe einer längshomogenen Leitung. Die Analyse der Wellenanpassung an den Transceiver ist in dieser Arbeit nicht gemacht worden. Die Theoretischengundlagen wurden jedoch bereits beschrieben. \\

Ein weiterer interessanter Vergleich der Messresultate und der Theorie ergibt sich aus der Abstrahlcharakteristik des Funktionsmusters.\\
Aus der Theorie erwartet man eine torusförmige Abstrahlung radial um die Antenne. Weiter ist aus der Theorie bekannt, dass das E-Feld der Dipolantenne viel Energie im reaktiven Nahfeldspeichert, wenn Kunststoff sich im Nahfeld befindet. Die Simulationen zeigen ebenfalls, dass die Feldausbreiung nicht kreisrund ist, daher ist die Abbildung \ref{fig:Schnittgemessen} in Kapitel {sec:Messergebinsse} nicht verwunderlich. Sie zeigt Den Gewinn der Antenne in der xz-Ebene.\\
Aus der Abbildung \ref{fig:DipolEFerd} ist das E-Feld einer Dipol Antenne gezeigt. Je eine Nullstelle, des elektromagnetischen Feldes, ergibt sich in der Verlängerung der Dipolachse. Eine eindeutig Nullstelle lässt sich aus der Abbildung \ref{fig:Schnittgemessen} nicht lesen. Jedoch auffällig ist, dass genau entlang der positiven und negativen y-Achsen Ausrichtung eine Einbuchtung der Feldstärkte ersichtlich wird. Eine Erklärung für die starke Signaldämpfung entlang der positiven und negativen x-Achse in der xy-Ebene könnten wider bei den absorbierenden Eigenschaften des Geräts sein. Im Zentrum des $\lambda/2$-Dipols herrscht immer die kleinste Spannung. Ist der Dipol in Resonanz so tritt an der Speisestelle ein Kontenpunkt auf, das könnte ein Erklärung für die schwache Feldabstrahlung im Uhrsprung sein.\\
Aus den Simulationsdateien und dem qualitativen Abstrahldiagramm des Funktionsmusters, welches in Abbildung \ref{fig:3D Richtdiagramm} dargestellt ist, kann entnommen werden dass ein isotropes Abstrahlverhalten mit einer Kompaktantenne in einem mobilen Gerät kaum möglich ist. Jedoch ist ein quasi Kugelförmiges elektromagnetisches Feld um das Gerät möglich. Es die Feldstärke wird schwanken und es kann zu Verbindungsunterbrüche kommen. Das in dieser Arbeit gefunden Antennedesign hat jedoch den grossen Vorteil, dass es symmetrisch ist, das heisst auf die bis anhin verwendeten Balun kann verzichtete  werden. Weiter ist der gezeigte Wirkungsgrad mit rund 50 $\%$ im der Bluetooth Sendespektrum sehr viel grösser als der Wirkungsgrad der bisherigen Bluetooth Antenne.\\

Trotz der erreichten $S_{10}$-Forderung von -10 dB über einen Frequenzbereich von 2.4 GHz bis 2.5 GHz und des hohen Abststrahleffizienz von rund 50 $\%$ empfehle ich, das Antennendesign weiter zu optimieren und so eine höhere Abstahleffizienz zu erreichen und damit verbunden den Wirkungsradius der Antenne zu vergrössern. An der Richtcharakteristik wird sich jedoch nur mit einer Steigerung der Anpassung und des Wirkungsgrad nichts ändern. 

 
