\newpage

\subsection{Loop Antenne}\label{sec:LoopAntenneTheorie}

%%%%%%%%%%%%%%%%%%%%%%%%%%%%%%%%%%%%%%%%%%%%%%%%%%%%%%%%%%%%%%%%%%%
\begin{figure}[h]
	\centering
	\includegraphics[width=7cm]{content/bilder/Loop_EMANT_S45.pdf}%
	\caption{Loop Antenne}
	\label{LoopAntenne}
\end{figure}
%%%%%%%%%%%%%%%%%%%%%%%%%%%%%%%%%%%%%%%%%%%%%%%%%%%%%%%%%%%%%%%%%%%


Wird eine kurze, kreisförmige Stromschleife mit dem Radius $a<<\lambda$ von einem Strom $Ie^{j\omega t}$ durchflossen,  kann in guter Näherung eine konstante Stromverteilung I entlang der Schlaufe angenommen werden,
Die Koordinaten eines Punktes auf der Stromschleife sind gegeben mit
\begin{eqnarray}
x’ &=&a \cos(\psi)\\
y’ &=&a \sin(\psi)\\
z’ &=&0
\end{eqnarray}


Der Abstand  a entspricht dem Radius  aus dem Zentrum zur Stromschleife. Die Stromschleife ist in  Abbildung \ref{{LoopAntenne}} ersichtlich. Somit kann ein Stromelement auf der Schleife beschrieben werden als:
\begin{equation}
I dl= Ia(- \vec e_{x}sin(\psi)+\vec e_{y}cos(\psi))d\psi
\end{equation}
 
%EMANT Joss Seit 45
Die Stromverteilung führt zu einem Abstrahlen von Elektromagnetischen Wellen.
%Die Gewichtungsfaktoren findet man mit Hilfe von (116 Joss) und (117 Joss) zu
%Emant Joss Seite 46
%Emant Joss Seite 46
Da eine Integration über $2\pi $ einer Kreisfunktion Null ergibt, findet man direkt 
\begin{equation}
a(\theta, \phi) =0
\end{equation}
\todo{Was sthet hier?}
Nimmt man  an, dass $ka$ klein ist, so kann der folgende Term vereinfacht werden.
\begin{equation}
a(\theta)=sin(ka sin(\theta)cos(\psi))=ka sin(\theta)cos(\psi)
\end{equation}
dank der Fereinfachung kann xxxx als geschrieben werden.
\begin{equation}
a(\theta)=j(\pi a^{2}I)(k sin \theta)
\end{equation}
%$a(theta, phi)=j$... oder Ellito 2.31 oder Joss Seite 46.
\todo{neu schreiben Loop}
Das Fernfeld ist  ist  $\varphi$ polarisiert.  Die Leistungsdichte gewinnt man mit 
\todo{Ebenen Loop}
%(Joss 118) 
\begin{equation}
\vec P(\theta,\phi)=\frac{1}{2}Re(\vec E x \vec H^*)
\end{equation}
zu
%Joss EMANT P(theta,phi)=....Ellito 2.32
\begin{equation}
P(\theta)=\frac{(ka)^{4}I{2}\eta}{32r^{2}}sin^{2}\theta
\end{equation}
Im Vergleich mit dem kurzen Dipol erzeugt die kleine Stromschleife ein vergleichbares Richtdiagramm. Das Fernfeld des kurzen Dipols ist jedoch vertikal $\vartheta$ polarisiert. Das bedeutet, dass das Abstrahlverhalten um 90Grad verschoben ist. Integriert man die Leistungsdichte über eine Kugeloberfläche mit dem Radius r  auf und setzt sie der abgegebenen Leistung mit $1/2 I^{2}Rrad $ der zugeführten Zweidrahtleitung gleich, so gewinnt man $R_{rad}$ mit 
\begin{equation}
R_{rad} = 320\pi^{6} (a/\lambda)^{4}\label{eq:RradLoop}
\end{equation}
%ELLITOH 2.33
Beispiel:\\
Wenn $a/\lambda = 0.03$ ist, dann wird der $R_{rad} = 0.25 Ohm$. Als Vergleich mit dem kurzen Dipol mit der Länge $2l=\lambda= 0.06$ führt das zu einem Strahlungswiderstand $R_{rad}$ von 0.7 Ohm.  Der Abstrahlwiderstand $R_{rad}$ einer kleinen Stromschleife kann um den Faktor $n^{2}$ erhöht werden, wenn n die Anzahl der sehr eng aneinander liegenden Wicklungen der Stromschleife sind. 



%%%%%%%%%%%%%%%%%%%%%%%%%%%%%%%%%%%%%%%%%%%%%%%%%%%%%%%%%%%%%%%%%%%

\begin{figure}[h!]
\begin{center}
\begin{tikzpicture}
	\draw (0,3) node at (0.5,0.5) {xz Ebene} -- (10,3);%Fadenkreuz horizontal
	\draw (5,0) -- (5,6);%Fadenkreuz vertikal
	\draw (3,3) circle (2cm);%linker Kreis
	\draw (7,3) circle (2cm);%rechter Kreis
	\draw (4.5,3) circle (0.2cm);%linker kleiner Kreis
	\draw (5.5,3) circle (0.2cm);%rechter kleiner Kreis
	\draw (4.5,3.2) -- (5.5,3.2);%Verbindung horizontal oben
	\draw (4.5,2.8) -- (5.5,2.8);%Verbinung horizontal unten
	\node[draw] at (5,6.5) {$\vartheta=0$};
	\node[draw] at (8,5.5) {$E_{\varphi}(\vartheta)$};

\end{tikzpicture}
\end{center}
\caption{E Feld einer Loop Antenne in der xz Ebene}
\label{DipolEFerd}
\end{figure}
%%%%%%%%%%%%%%%%%%%%%%%%%%%%%%%%%%%%%%%%%%%%%%%%%%%%%%%%%%%%%%%%%%%