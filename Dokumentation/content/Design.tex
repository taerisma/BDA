\newpage
\section{Design}
\subsection{Anforderungen}
Das Design des Antennensystems wird für einen Anwendungsfall im Freiraum dimensioniert. Die Distanz zwischen Sender und Empfänger soll 10 Meter betragen. Das Übertragungsmedium ist Luft, kann aber idealisiert als Vakuum angenommen werden. Das System soll isotrop abstrahlen und der Gewinn der Empfangsantenne kann mit einem Faktor  1 angenommen werden. Die Antenne soll symmetrisch gespiesen werden und im 2.4 GHz ISM Band arbeiten. Als Quelle dient ein Bluetooth Low Energie Texas Instruments CC2541 Chip mit 0 dBm als Sendeleistung. Als Designkriterien wird eine S11 Dämpfung von 10 dB und eine Reserve von 6 dB dienen.


%\subsection{Technische Spezifikationen und Anforderungsliste}
%%\todo{Anforderungskatalogs mit Fest-, Mindest- \& Wunschforderungen}
%\begin{itemize}
%\item Geräte Connect 1
%\item Materialien des Gehäuse ABS Kusnstoff
%\item Volumen des Antennensystems
%\item Wirkungsradius 10m im Freiraum
%\item Richtcharakteristik isotroph
%\item Polarisation linear
%\item Antennen Wirkungsgrad ist zubestimmen
%\item Antennen Gewinn gleich wir der Abstrahl Wirkungsgrad
%\item minimaler Empfangspegel am Transceivers
%\item Transceivers Baustein Texas Instruments CC2541
%\item Sendeleistung
%\item $S_{11} \leq$ 10 dB
%\end{itemize}

\begin{tabular}{l|c|c|c|c}
\hline 
Nr. & Anforderung & Beschreibung & Wert & nicht erfüllt \\ 
\hline 
\hline 
001 & f & ISM Frequenzbereich  & 2.4-2.5 GHz & \\ 
\hline 
002 & f & Handgerät lxbxh & lxbxh &   \\ 
\hline 
003 & f & symmetrische Speisung des Antennensystems &  \\ 
\hline 
004 & f & Reflexionskoeffizient der Antenne S11 & 10dB & \\ 
\hline 
005 & f & Funkdistanz, Arbeitsradius & 10m &  \\ 
\hline 
006 & f & Linkbudget Reserve & 6dB &  \\ 
%• &  &  &  &  \\ 
%\hline 
%• &  &  &  &  \\
%\hline 
%• &  &  &  &  \\ 
%\hline 
%• &  &  &  &  \\ 
%\hline 
%• &  &  &  &  \\ 
%\hline
%• &  &  &  &  \\ 
%\hline 
%• &  &  &  &  \\
%\hline 
\end{tabular} 

\subsection{Analyse mit bekannten Modellen}
\subsection{Neue Design Ansätze}