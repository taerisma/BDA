\newpage
\section*{Zusammenfassung}

Gleitschirmpiloten greifen zur Orientierung in der Luft auf technische Hilfsmittel zurück. Beispielsweise kommen Variometer zur Höhenbestimmung sowie GPS-Geräte zur Positionsbestimmung zum Einsatz. Die Firma Flytec stellt seit über 30 Jahren verschiedene Fluginstrumente für die Tuchfliegerei her. Diese Instrumente vernetzen eine Reihe von Sensoren, sodass dem Piloten verschiedene Parameter übermittelt werden können. Um eine vielseitige Datenkommunikation sicher zu stellen, wird von der \glqq Connect 1 \grqq Geräteserie ein Bluetooth Netzwerk und ein WiFi Netzwerk zur Verfügung gestellt. Um das Bluetooth Netzwerk in Zukunft mit einem Smartphone zu verbinden, muss das bisherige Antennensystem verbessert werden.
Diese Arbeit untersucht ein bestehendes 2.4 GHz \glqq near pilot network \grqq , welches in der \glqq Connect 1 \grqq Gerätefamilie der Firma Flytec AG zur Anwendung kommt. 
In diesem  Gerät kommen drei Antennensysteme zum Einsatz, welche alle in frei zugänglichen ISM Frequenzbereichen arbeiten. Die Systeme beeinflussen sich daher gegenseitig. Die Wahl der Antennen und deren Positionierung beeinflusst das Abstrahlverhalten signifikant und ist für eine einwandfreie Funktion äusserst wichtig. In dieser Arbeit wird ein technisch realisierbares Design für die \glqq Bluetooth Low Energy \grqq Antenne gesucht.  
Nach der Überprüfung von zwei symmetrisch gespiesenen Antennenkonzepten in der Entwicklungsphase mit entsprechenden Simulationen im EMPIRE XPU, zeigte sich die Dipolantenne für den Einsatz in der  \glqq Connect 1 \grqq Serie am Vielversprechendsten.
 Daher wurde das Dipolantennenkonzept im Antennenentwicklungsprozess genauer untersucht und mögliche Designvarianten für den Einbau in die Flytec \glqq Connect 1 \grqq Geräteserie
  erarbeitet, mit dem Ziel ein möglichst optimales Abstrahlverhalten zu finden.\\
  \colorbox{yellow}{\parbox[t]{\textwidth}{Es hat sich gezeigt, dass eine Dipolantenne mit einer Breite von xx mm un deiner Länge von yy mm 
   ein Abstrahlverhalten zeigt, welches den Anforderungen sehr nache kommt. 
   Die Abstrahleffizienz ist und die Antennenimpedanz mit $Z_{ant}\ = \ (a+jb)\Omega$. \\
   $Z_{ant}$ stellt dank der Anpassung mit einen em Koaxilkabel der Länge zzz mm eine ideale Lösung für den Einsatz in der  \glqq Connect 1\grqq Serie. }}

