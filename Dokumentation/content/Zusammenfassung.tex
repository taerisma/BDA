\newpage
\section*{Zusammenfassung}

Die Firma Flytec AG stellt seit über 30 Jahren verschiedene Fluginstrumente für die Tuchfliegerei her. Das „Connect 1“ Fluggerät beinhaltet ein Bluetooth-Netzwerk, um eine Datenverbindung mit einem Smartphone zu ermöglichen. Hierfür muss die bisher verwendete Antenne verbessert werden. Die Bachelorarbeit „Entwurf einer Kompaktantenne“ von Pascal Schantl (06. Juni 2014) zeigt, dass die Wahl der Antennenart und die Positionierung der Antenne das Abstrahlverhalten derselben sowie des gesamten Systems signifikant beeinflussen. In dieser Arbeit soll unter Berücksichtigung dieser Gesichtspunkte ein technisch realisierbares Funktionsmuster für die zukünftige 2.4 GHz Antenne entwickelt werden.\\ 
In einem ersten Schritt werden durch Erarbeiten der technischen Grundlagen mögliche Antennenarten für das zu entwickelnde Funktionsmuster eruiert. Im nächsten Schritt der Simulationsphase werden die symmetrisch gespeiste Loop Antenne und Dipol Antenne mit entsprechenden Simulationen im EMPIRE XPU genauer untersucht. In der Entwicklungsphase werden verschiedene Designvarianten für den Einbau in das Fluginstrument „Connect 1“ für die Antenne mit dem vielversprechendsten Potential simuliert, mit dem Ziel ein möglichst optimales Abstrahlverhalten zu finden.\\
In der Simulationsphase zeigte sich die Dipol Antenne für den Einsatz im „Connect 1“ Fluggerät am Vielversprechendsten, weshalb vier verschiedene Varianten einer Dipol Antenne in der Designphase weiterverfolgt wurden. Es hat sich gezeigt, dass eine Dipol Antenne mit einer Breite von 3 mm und einer Länge von 50.25 mm ein Abstrahlverhalten zeigt, welches den Anforderungen sehr nahe kommt. Mit einer Antennenimpedanz von (30+j4) $\Omega$ resultiert eine gemessene Abstrahleffizienz von 49 $\%$ bei der  Zielfrequenz von 2.45 GHz.\\
Die Dipol Antenne mit oben genannten Charakteristika hält das vorgegebene Antennenvolumen ein. Ebenso wird die gewünschte Sendebandbreite erreicht. Zudem kann durch die symmetrische Antenne auf den bisher verwendeten Balun verzichtet werden. Die Abstrahleffizienz konnte im Vergleich zur bisherigen Antenne deutlich verbessert werden, sie erreicht jedoch noch nicht die simulierten Werte. Eine weitere Erhöhung derselben könnte durch Optimierung der Antennenstruktur mit folglich verbesserter Anpassung an den Transceiver erreicht werden, aus zeitlichen Gründen konnte dies im Rahmen der vorliegenden Arbeit nicht weiter vertieft werden.




