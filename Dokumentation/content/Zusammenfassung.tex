\newpage
\section*{Zusammenfassung}

Gleitschirmpiloten greifen zur Dokumentation ihrer Flugrouten auf technische Hilfsmittel zurück. Beispielsweise kommen GPS-Geräte zur Positionsbestimmung zum Einsatz. Die Firma Flytec AG stellt seit über 30 Jahren verschiedene Fluginstrumente für die Tuchfliegerei her. Diese Instrumente vernetzen eine Reihe von Sensoren, sodass dem Piloten verschiedene Parameter übermittelt werden können. Um eine vielseitige Datenkommunikation sicher zu stellen, wird von der \glqq Connect 1\grqq \  Geräteserie ein Bluetooth- und ein WiFi-Netzwerk zur Verfügung gestellt. Damit das Bluetooth-Netzwerk in Zukunft mit einem Smartphone verbindet werden kann, muss das bisherige Antennensystem verbessert werden. Zu diesem Zweck wird im Rahmen dieser Arbeit ein Funktionsmuster für die bestehende 2.4 GHz Antenne erarbeitet. Diese kommt  neben weiteren Antennen, welche alle im frei zugänglichen ISM Frequenzbereichen arbeiten, zum Einsatz. Dass sich die verwendeten Antennen gegenseitig beeinflussen, zeigte die Bachelorarbeit mit dem Titel „Entwurf einer Kompaktantenne“ von Pascal Schantel.
Allgemein bekannt ist, dass die Wahl der Antennen und deren Positionierung das Abstrahlverhalten der einzelnen Antennen und des gesamten Systems signifikant beeinflussen. In dieser Arbeit soll ein technisch realisierbares Funktionsmusters für die zukünftige 2.4 GHz Antenne gesucht werden.  Nach der Überprüfung von zwei symmetrisch gespiesenen Antennenkonzepten in der Entwicklungsphase mit entsprechenden Simulationen im EMPIRE XPU, zeigte sich die Dipolantenne für den Einsatz am Vielversprechendsten. Daher wurde das Dipolantennenkonzept im Antennenentwicklungsprozess genauer untersucht und mögliche Designvarianten für den Einbau in die Flytec AG Geräte erarbeitet, mit dem Ziel ein möglichst optimales Abstrahlverhalten zu finden.\\
  \colorbox{yellow}{\parbox[t]{\textwidth}{Es hat sich gezeigt, dass eine Dipolantenne mit einer Breite von xx mm un einer Länge von yy mm 
   ein Abstrahlverhalten zeigt, welches den Anforderungen sehr Nahe kommt. 
   Die Abstrahleffizienz ist xxxxx.\\
   $Z_{ant}$ stellt dank der Phasenverschiebung  mit einem  Koaxialkabel der Länge zzz mm eine ideale Anpassung  an den Transceiver CC2541 von Texas Instruments des Geräts dar.}}\\


