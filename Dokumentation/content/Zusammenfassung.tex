\section*{Zusammenfassung}

Gleitschirmpiloten greifen zur Orientierung in der Luft auf technische Hilfsmittel zurück. Beispielsweise kommen Variometer zur Höhenbestimmung sowie GPS-Geräte zur Positionsbestimmung zum Einsatz. Die Firma Flytec stellt seit über 30 Jahren verschiedene Fluginstrumente für die Tuchfliegerei her. Diese Instrumente vernetzen eine Reihe von Sensoren. Um eine einwandfreie Datenkommunikation sicher zu stellen, wird ein Bluetooth Netzwerk und ein WiFi Netzwerk zur Verfügung gestellt. Um das Bluetooth Netzwerk in Zukunft mit einem Smartphone zu verbinden, muss das Antennensystem verbessert werden.
Diese Arbeit untersucht ein bestehendes 2.4 GHz „Bluethooth Low Energie Netzwerk“, welches in der „Connect 1“ Gerätefamilie der Firma Flytec AG zur Anwendung kommt. In diesem  Gerät kommen drei Antennensysteme zum Einsatz, welche alle in frei zugänglichen ISM Frequenzbereichen arbeiten. Die Systeme beeinflussen sich daher gegenseitig. Die Wahl der Antennen und deren Positionierung beeinflusst das Abstrahlverhalten signifikant und ist für eine einwandfreie Funktion äusserst wichtig. In dieser Arbeit wird ein technisch realisierbares Design für die „Bluetooth Low Energie“ Antenne gesucht. Die dafür notwendige Antennentheorie wird beschrieben. Daher soll ein technisch realisierbares Design für eine Bluetooth Low Energie Antenne entwickelt werden. Die Antenne soll in einem Handgerät zur Anwendung kommen. Nach der Überprüfung von zwei Konzepten in der Entwicklungsphase mit entsprechenden Simulationen im Empire Xccel, war das Dipol Konzept am vielversprechendsten. Dieses wurde im Antennenentwicklungsprozess genauer untersucht und mögliche Designvarianten für den Einbau in die Flytec „Connect 1“ Geräteserie untersucht, mit dem Ziel ein möglichst optimales Abstrahlverhalten zu finden.
\todo{Entscheidungsstelle etwas hervorheben}