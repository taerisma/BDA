\newpage
\section*{Zusammenfassung}

Die Firma Flytec AG stellt seit über 30 Jahren verschiedene Fluginstrumente für die Tuchfliegerei her. Das \glqq Connect 1\grqq\ Fluggerät beinhaltet ein Bluetooth-Netzwerk, um eine Datenverbindung mit einem Smartphone zu ermöglichen. Hierfür muss die bisher verwendete Antenne verbessert werden. Die Bachelorarbeit „Entwurf einer Kompaktantenne“ von Pascal Schantel zeigt, dass die Wahl der Antenne und deren Positionierung das Abstrahlverhalten derselben sowie des gesamten Systems signifikant beeinflussen. In dieser Arbeit soll unter Berücksichtigung dieser Gesichtspunkte ein technisch realisierbares Funktionsmusters für die zukünftige 2.4 GHz Antenne gesucht werden.\\
Nach Überprüfung zweier symmetrisch gespeister Antennen in der Entwicklungsphase mit entsprechenden Simulationen im EMPIRE XPU soll die vielversprechendere Antenne in einem  Entwicklungsprozess genauer untersucht und mögliche Designvarianten für den Einbau in die Flytec AG Geräte erarbeitet werden, mit dem Ziel ein möglichst optimales Abstrahlverhalten zu finden. Hierfür werden die verschiedenen Funktionsmuster nach initialer Simulation praktisch gefertigt und unter den Projektbedingungen ausgemessen.\\
In der Entwicklungsphase zeigte sich die Dipol Antenne für den Einsatz in Fluginstrument am Vielversprechendsten. Daher wurde das Dipolantennenkonzept weiter verfolgt. Es hat sich gezeigt, dass eine Dipolantenne mit einer Breite von 3 mm und einer Länge von 50.25 mm ein Abstrahlverhalten zeigt, welches den Anforderungen sehr nahe kommt. Mit einer Antennenimpedanz von (30+j4) $\Omega$ resultiert eine gemessene Abstrahleffizienz von 49 $\%$ bei der Zielfrequenz von 2.45 GHz. \\
Die Dipol Antenne mit oben genannten Charakteristika hält das vorgegebene Antennenvolumen ein. Ebenso wird die gewünschte Sendebandbreite erreicht. Zudem kann durch die symmetrische Antenne auf den bisher verwendeten Balun verzichtet werden. Die Abstrahleffizienz konnte im Vergleich zur bisherigen Antenne deutlich verbessert werden, erreicht jedoch noch nicht die simulierten Werte. Eine weitere Erhöhung derselben könnte am ehesten durch die Optimierung der Antennenstruktur mit folglich verbesserter Anpassung an den Tranceivers erreicht werden. 



