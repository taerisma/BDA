\subsection{Ausgangslage}
Die Firma Flytec vernetzt Sensoren und Fluginstrumente drahtlos mit einem  „near pilot network“. Ihre Fluginstrumente stellen für die Piloten von Tuchfliegeren ein grosse Hilfe dar. Die Navigation in der Luft stellte selbst für erfahrene Piloten eine grosse Herausforderung dar. Für ein Fluginstrument der „Connect 1“ Serie  soll eine integrierte Kompaktantenne für das ISM Band 2.4 GHz entworfen werden. 

\subsection{Problematik}
Bestandteil dieser Arbeit ist das Entwickeln einer 2.4 GHz Bluetooth Antenne. Diese soll die Kommunikation mit einem Smartphone sicher stellen. Die bisher verwendete Antenne hat nie ihre gewünschte Wirkung gezeigt. Die Antenne wird von dem Bluetooth Low Energie Chip CC2541 der Firma Texas Instruments getrieben. In dem Gerät befinden sich mehrere Antennen die im selben Frequenzbereich arbeiten, die gegenseitige Koppelung der Antennen ist gross und sowohl  von der Wahl der Antennen sowie der Positionierung abhängig. Es soll eine möglichst optimale symmetrisch gespiesesnes Antennenkonzept für die Bluetoothverbindung erarbeitet werden. Das heisst auf die bis anhin verwendeten Baluns soll in Zukunft verzichtet werden.
\subsection{Fragestellung}
Für ein Fluginstrument der „Connect 1“ Serie  soll eine integrierte Kompaktantenne für das ISM Band 2.4 - 2.5 GHz entworfen werden, mit dem Ziel zukünftig Daten auf eine Smartphone Applikation zu übertragen, um dem Piloten mit den aktuellen Flugdaten über das Smartphone zu versorgen. 

Durch das Einarbeiten in die Eigenschaften elektrisch kleiner Antennen soll das Wissen für das Design eines Funktionsmuster  im 2.4 ISM Band arbeitenden Antennensystems erlangt werden. Das Design soll mit dem Empire Xccel Softwaretool simuliert. Iterativ soll eine symmetrisch gespiesene Kompaktantenne gefunden werden. Die Antenne soll an den komplexen Ausgangswiderstand des Bluetooth CC2541 Chip angepasst werden. Das simulierte Antennensystem soll gefertigt und ausgemessen werden. Die vorliegenden Antennenparameter sollen mit den simulierten Werten verglichen und bewertet werden. Ein Fazit über das erarbeitete Funktionsmuster soll den zukünftigen Einsatz in der „Connect 1“ darlegen.
\subsection{Ziele der Arbeit}
Es soll ein Funktionsmuster für eine integrierte symmetrisch gespiese-ne Antenne für das ISM 2.4 GHz Band hergestellt werden. Über diese Antenne sollen in Zukunft die Geräte der „Connect 1“ Serie über Bluetooth mit einem Smartphone kommunizieren lassen. 

Auf der Basis der Theorie der elektrisch kleinen Antennen wird ein Entwurf für ein Antennensystem im 2.4 GHz Band designed. Der Entwurf wird simuliert und dokumentiert. 
Ein Anpassnetzwerk muss für die komplexe Ausgangsimpedanz (70 +j30 Ohm bei 2.440 GHz) des Transsivers von Texas Instruments soll dimensioniert, simuliert, hergestellt und ausgemessen werden.
Der simulierte Entwurf des gesamten Antennensystems wird produziert und dient als Funktionsmuster. Das Abstrahlverhalten des Funktionmusters muss gemessen und dokumentiert werden.
Abweichungen zwischen der Simulation und den Messresultaten sollen do-kumentiert und bewertet werden.
\subsection{Methodik}
Diese Bachelor Arbeit beschreibt den Design Prozess eines 2.4 GHz Kom-pakt Antennen Systems. Es beinhaltet die Studien von Kompakt Antennen sowie deren Abstrahlverhalten. Ein Vorprojekt, welches verschiedene mögliche Antennenkonzepte prüft. Aus diesem Vorprojekt wird das vielversprechendste Konzept ausgewählt und für den Einsatz in die Geräte Serie „Connect 1“ der Firma Flytec AG optimiert. Dieser Prozess ist von Simulationen und vom erstellen und Ausmessen von Funktionsmustern begleitet. Die Erkenntnisse aus den Messungen und den Vergleichen aus Theorie und Praxis werden im einem Fazit zusammengefasst und es soll das weitere Vorgehen für die Firma Flytec dokumentiert werden. Für diese Bachelorarbeit stehen 15 Wochen zur Verfügung. Diese Zeit wird in die folgenden vier Phasen eingeteilt:
\begin{itemize}
	\item Recherche- und Theoriephase
	\item Designphase
	\item Prototyping 
	\item Dokumentation des Engeneeringmodels
\end{itemize}

\subsection{Aufbau der Arbeit}
Da muss noch was rein



