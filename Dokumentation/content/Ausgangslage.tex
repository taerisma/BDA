\section{Ausgangslage}
Die Firma Flytec vernetzt Sensoren und Fluginstrumente drahtlos mit einem sogenannten \glqq near pilot network\grqq. Hierbei werden Messdaten über Bluetooth oder WiFi an das Smartphone des Piloten übermittelt. Dafür wird in den Fluggeräten eine Bluetooth-Antenne benötigt. 

\section{Problematik}
Die bisher verwendete Antenne weist nicht den vollen Funktionsumfang auf, deshalb soll im Rahmen dieser Arbeit eine integrierte Kompaktantenne für das ISM Band 2.4 GHz entworfen werden, welche diese Datenübermittlung übernimmt. Bis anhin wurde ein \glqq Bluetooth Low Energy\grqq Chip CC2541 der Firma Texas Instruments mit einem symmetrischen Quellenausgang als Antennenquelle mit einem asymmetrischen  Antennensystem kombiniert, was ein Balun zur Impedanzumwandlung notwendig machte.  Aus diesen Gründen soll ein symmetrisch gespiesenes Antennenkonzept für die Bluetooth-Verbindung entwickelt werden, sodass auf die bis anhin verwendeten Baluns verzichtet werden kann. Zudem befinden sich in den Fluggeräten mehrere Antennen, welche im gleichen Frequenzbereich arbeiten. Wie die Vorläuferarbeit zeigte, führt dies zu Wechselwirkungen zwischen den Antennen und damit zur Beeinträchtigung des Abstrahlverhaltens. Da diese Wechselwirkungen von der Wahl der Antennen sowie deren Positionierung abhängig sind, sollen diese ebenfalls optimiert werden. 

\section{Fragestellung}
Aufgrund der Ausgangslage und der oben beschriebenen Problematik ergibt sich folgende Fragestellung: 
Wie lässt sich die von der Firma Flytec AG verwendete Bluetooth-Antenne durch Wahl der Antennenart und deren Positionierung optimieren, um eine hohe Abstrahleffizienz sowie eine isotrope Abstrahlcharakteristik zu erreichen?



\section{Ziele der Arbeit}

Um oben genannte Fragestellung beantworten zu können, soll durch das Einarbeiten in die Eigenschaften elektrisch kleiner Antennen das Wissen für die Anfertigung eines Funktionsmusters einer im 2.4 GHz ISM Band arbeitenden Antenne erlangt werden. Initial soll basierend auf der Therorie iterativ ein Design für eine symmetrisch gespiesene Kompaktantenne gefunden werden, welches den Anforderungen entspricht. Die verschiedenen Lösungsansätze werden anschliessend mit dem EMPIRE XPU Softwaretool simuliert, indem die einzelnen Antennenparameter variiert werden. Das vielversprechendste Design wird als Funktionsmuster hergestellt und gegebenenfalls an den komplexen Ausgangswiderstand des Bluetooth CC2541 Chip angepasst. Das gefertigte Antennensystem soll ausgemessen und die vorliegenden Antennenparameter mit den simulierten Werten verglichen werden. Ein Fazit über das erarbeitete Funktionsmuster soll die Arbeit abschliessen.

\section{Methodik}
Für die Erarbeitung eines Funktionsmusters einer 2.4 GHz Kompaktantenne werden die für diese Bachelorarbeit zur Verfügung stehenden 15 Wochen in folgende drei Phasen unterteilt: 
\begin{itemize}
	\item Recherche- und Theoriephase
	\item Design- und Simulationsphase 
	\item  Erstellung und Verifikation des Funktionsmusters
\end{itemize}
Nach einer initialen Phase des Theoriestudiums werden in einem Vorprojekt die beiden bekannten symmetrischen Antennenkonzepte sowie die verschiedenen möglichen Ausführungen derselben geprüft. Das vielversprechendste Konzept wird in der Designphase für den Einsatz in den Fluggeräten der Flytec AG optimiert. Dieser Prozess ist von Simulationen sowie dem  Erstellen und Ausmessen des erstellten Funktionsmusters begleitet. Die Erkenntnisse aus den Messungen und den Vergleichen aus Theorie und Praxis werden im einem Fazit zusammengefasst sowie das weitere Vorgehen für die Firma Flytec dokumentiert werden.



