\section{Ausgangslage}
Die Firma Flytec vernetzt Sensoren und Fluginstrumente drahtlos mit einem sogenannten \glqq near pilot network\grqq. Hierbei werden Messdaten über Bluetooth oder WiFi an das Smartphone des Piloten übermittelt. Dafür wird in den Fluggeräten eine Bluetooth-Antenne benötigt. 

\section{Problematik}
Die bisher verwendete Antenne weist nicht den vollen Funktionsumfang auf, deshalb soll im Rahmen dieser Arbeit eine integrierte Kompaktantenne für das ISM Band 2.4 GHz entworfen werden, welche diese Datenübermittlung übernimmt. Bis anhin wurde ein \glqq Bluetooth Low Energy\grqq Chip CC2541 der Firma Texas Instruments mit einem symmetrischen Quellenausgang als Antennenquelle mit einem asymmetrischen  Antennensystem kombiniert, was ein Balun zur Impedanzumwandlung notwendig machte.  Aus diesen Gründen soll ein symmetrisch gespiesenes Antennenkonzept für die Bluetooth-Verbindung entwickelt werden, sodass auf die bis anhin verwendeten Baluns verzichtet werden kann. Zudem befinden sich in den Fluggeräten mehrere Antennen, welche im gleichen Frequenzbereich arbeiten. Wie die Vorläuferarbeit zeigte, führt dies zu Wechselwirkungen zwischen den Antennen und damit zur Beeinträchtigung des Abstrahlverhaltens. Da diese Wechselwirkungen von der Wahl der Antennen sowie deren Positionierung abhängig sind, sollen diese ebenfalls optimiert werden. 

\section{Fragestellung}
Aufgrund der Ausgangslage und oben beschriebenen Problematik ergibt sich folgende Fragestellung: 
Wie lässt sich die von der Firma Flytec AG verwendete Bluetooth-Antenne durch Wahl der Antennenart und deren Positionierung optimieren, um eine isotrope Abstrahlcharakteristik sowie eine hohe Effizienz zu erreichen?
Ein entsprechendes Funktionsmuster der integrierten Kompaktantenne für das ISM Band 2.4 GHz soll entworfen werden, um die Simulationswerte zu überprüfen.


Durch das Einarbeiten in die Eigenschaften elektrisch kleiner Antennen soll das Wissen für ein Design eines Funktionsmusters  im 2.4 GHz ISM Band arbeitenden Antennensystems erlangt werden. Das Design wird mit dem EMPIRE XPU Softwaretool simuliert. Iterativ soll eine symmetrisch gespiesene Kompaktantenne gefunden werden. Die Antenne soll an den komplexen Ausgangswiderstand des Bluetooth CC2541 Chip angepasst werden. Das simulierte Antennensystem soll gefertigt und ausgemessen werden. Die vorliegenden Antennenparameter sollen mit den simulierten Werten verglichen und bewertet werden. Ein Fazit über das erarbeitete Funktionsmuster soll den zukünftigen Einsatz in der \glqq Connect 1 \grqq Serie darlegen.
\section{Ziele der Arbeit}
Es soll ein Funktionsmuster für eine integrierte, symmetrisch gespiesene Antenne für das ISM 2.4 GHz Band hergestellt werden. Über diese Antenne sollen in Zukunft die Geräte der \glqq Connect 1 \grqq Serie über das Bluetooth-Protokoll mit einem Smartphone kommunizieren. 

Auf der Basis der Theorie der elektrisch kleinen Antennen wird ein Entwurf für ein Antennensystem im 2.4 GHz Band designed. Der Entwurf wird simuliert und dokumentiert. 

Ein Anpassungsnetzwerk  für die komplexe Ausgangsimpedanz (70 +j30 Ohm bei 2.440 GHz) des Transceivers CC2541 von Texas Instruments soll dimensioniert und beschrieben werden.
Der simulierte Entwurf des gesamten Antennensystems wird produziert und dient als Funktionsmuster. Das Abstrahlverhalten des Funktionsmusters wird gemessen und dokumentiert.
Abweichungen zwischen der Simulation und den Messresultaten sollen dokumentiert und bewertet werden.
\section{Methodik}
Diese  Arbeit beschreibt den Design Prozess eines 2.4 GHz Kompaktantennensystems. Es beinhaltet die Studien von Kompaktantennen sowie deren Abstrahlverhalten. Daraus ergibt sich ein  Vorprojekt, welches zwei mögliche symetrische Antennenkonzepte prüft. Aus diesem Vorprojekt wird das vielversprechendste Konzept ausgewählt und für den Einsatz in die Geräte Serie \glqq Connect 1 \grqq der Firma Flytec AG optimiert. Dieser Prozess ist von Simulationen, dem  Erstellen und Ausmessen der Funktionsmuster begleitet. Die Erkenntnisse aus den Messungen und den Vergleichen aus Theorie und Praxis werden im einem Fazit zusammengefasst. Das weitere Vorgehen für die Firma Flytec soll dokumentiert werden. Für diese Bachelorarbeit stehen 15 Wochen zur Verfügung. Diese Zeit wird in die folgenden drei Phasen eingeteilt:
\begin{itemize}
	\item Recherche- und Theoriephase
	\item Design- und Simulationsphase 
	\item Funktionsmuster Erstellung und Verifikation
\end{itemize}

\section{Aufbau der Arbeit}
Da muss noch was rein....\\
Punkteweise welche Kapitel hat das Dokument
\todo{Wie ist das Dokument aufgebaut}




