\subsection{Ausgangslage}
Die Firma Flytec vernetzt Sensoren und Fluginstrumente drahtlos mit einem  „near pilot network“. Ihre Fluginstrumente sind für die Piloten von Tuchfliegern eine grosse Hilfe. Die Navigation in der Luft stellt selbst für erfahrene Piloten eine grosse Herausforderung dar. Für die Fluginstrumente der „Connect 1“ Serie  soll eine integrierte Kompaktantenne für das ISM Band 2.4 GHz entworfen werden. 

\subsection{Problematik}
Ein Bestandteil dieser Arbeit ist das Entwickeln einer 2.4 GHz Bluetooth Antenne. Die Antenne soll die Kommunikation mit einem Smartphone sicher stellen. Die bisher verwendete Antenne hat nicht die gewünschte Wirkung gezeigt. Die Antenne wird von dem „Bluetooth Low Energie“ Chip CC2541 der Firma Texas Instruments getrieben. Im Gerät "Connect 1" \ befinden sich mehrere Antennen, die im selben Frequenzbereich arbeiten. Die gegenseitige Kopplung der Antennen ist gross und  stark von der Wahl der Antennen sowie deren Positionierung abhängig. Es soll ein  symmetrisch gespiesenes Antennenkonzept für die Bluetooth Verbindung erarbeitet werden. Auf die bis anhin verwendeten Baluns kann in Zukunft verzichtet werden, da die bisherige Bluetooth Antenne asymmetrisch war. 
\subsection{Fragestellung}
Für die Fluginstrumente der „Connect 1“ Serie  soll eine integrierte Kompaktantenne für das ISM Band 2.4 - 2.5 GHz entworfen werden mit dem Ziel, zukünftig Daten auf eine Smartphone Applikation zu übertragen. Dadurch werden die Piloten mit den aktuellen Flugdaten über das Smartphone  versorgt. Das neu zu designende Bluetooth Antennensystem soll eine isotrope Abstrahlcharakteristik und eine
 hohe Effizienz aufweisen. Die gegenseitige Kopplung mit den bestestehenden Antennen soll gering sein. 


Durch das Einarbeiten in die Eigenschaften elektrisch kleiner Antennen soll das Wissen für ein Design eines Funktionsmusters  im 2.4 GHz ISM Band arbeitenden Antennensystems erlangt werden. Das Design wird mit dem EMPIRE XPU Softwaretool simuliert. Iterativ soll eine symmetrisch gespiesene Kompaktantenne gefunden werden. Die Antenne soll an den komplexen Ausgangswiderstand des Bluetooth CC2541 Chip angepasst werden. Das simulierte Antennensystem soll gefertigt und ausgemessen werden. Die vorliegenden Antennenparameter sollen mit den simulierten Werten verglichen und bewertet werden. Ein Fazit über das erarbeitete Funktionsmuster soll den zukünftigen Einsatz in der „Connect 1“ Serie darlegen.
\subsection{Ziele der Arbeit}
Es soll ein Funktionsmuster für eine integrierte, symmetrisch gespiesene Antenne für das ISM 2.4 GHz Band hergestellt werden. Über diese Antenne sollen in Zukunft die Geräte der „Connect 1“ Serie über das Bluetooth-Protokoll mit einem Smartphone kommunizieren. 

Auf der Basis der Theorie der elektrisch kleinen Antennen wird ein Entwurf für ein Antennensystem im 2.4 GHz Band designed. Der Entwurf wird simuliert und dokumentiert. 

Ein Anpassungsnetzwerk  für die komplexe Ausgangsimpedanz (70 +j30 Ohm bei 2.440 GHz) des Transceivers CC2541 von Texas Instruments soll dimensioniert und beschrieben werden.
Der simulierte Entwurf des gesamten Antennensystems wird produziert und dient als Funktionsmuster. Das Abstrahlverhalten des Funktionsmusters wird gemessen und dokumentiert.
Abweichungen zwischen der Simulation und den Messresultaten sollen dokumentiert und bewertet werden.
\subsection{Methodik}
Diese  Arbeit beschreibt den Design Prozess eines 2.4 GHz Kompaktantennensystems. Es beinhaltet die Studien von Kompaktantennen sowie deren Abstrahlverhalten. Daraus ergibt sich ein  Vorprojekt, welches zwei mögliche symetrische Antennenkonzepte prüft. Aus diesem Vorprojekt wird das vielversprechendste Konzept ausgewählt und für den Einsatz in die Geräte Serie „Connect 1“ der Firma Flytec AG optimiert. Dieser Prozess ist von Simulationen, dem  Erstellen und Ausmessen der Funktionsmuster begleitet. Die Erkenntnisse aus den Messungen und den Vergleichen aus Theorie und Praxis werden im einem Fazit zusammengefasst. Das weitere Vorgehen für die Firma Flytec soll dokumentiert werden. Für diese Bachelorarbeit stehen 15 Wochen zur Verfügung. Diese Zeit wird in die folgenden drei Phasen eingeteilt:
\begin{itemize}
	\item Recherche- und Theoriephase
	\item Design- und Simulationsphase 
	\item Funktionsmuster Erstellung und Verifikation
\end{itemize}

\subsection{Aufbau der Arbeit}
Da muss noch was rein....\\
Punkteweise welche Kapitel hat das Dokument
\todo{Wie ist das Dokument aufgebaut}




