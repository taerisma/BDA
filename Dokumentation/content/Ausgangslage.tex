\section{Ausgangslage}
Die Firma Flytec vernetzt Sensoren und Fluginstrumente drahtlos mit einem  near pilot network. Für ein Fluginstrument der 7000er Serie  soll eine integrierte Loop Antenne für das ISM Band 2.4 - 2.5 GHz entworfen werden.\todo{Quelle der Aufgabenstellung einfügen}

\subsection{Fragestellung}
Durch das Einarbeiten in die Eigenschaften elektrisch kleiner Antennen soll das Wissen für das Design eines Funktionsmuster  im 2.4-2.5GHz ISM Band arbeitenden Antennensystems erlangt werden. Das Design soll mit dem Empire XCcel Softwaretool simuliert. Iterativ soll eine symmetrisch gespiesene Loop Antenne gefunden werden. Die Antenne soll an den komplexen Ausgangswiderstand des Transceiverbausteins angepasst werden. Das simulierte Antennensystem soll gefertigt und ausgemessen werden. Die vorliegenden Antennenparameter sollen mit den simulierten Werten verglichen und bewertet werden.  
\subsection{Technische Spezifikationen und Anforderungsliste}
\todo{Anforderungskatalogs mit Fest-, Mindest- \& Wunschforderungen}
\begin{itemize}
\item Geräte 7030
\item 
\item Materialien des Gehäuse ABS Kusnstoff
\item Volumen des Antennensystems
\item Wirkungsradius 10m im Freiraum
\item Richtcharakteristik isotroph
\item Polarisation linear
\item Antennen Wirkungsgrad ist zubestimmen
\item Antennen Gewinn gleich wir der Abstrahl Wirkungsgrad
\item minimaler Empfangspegel am Transceivers
\item Transceivers Baustein Texas Instruments CC2541
\item Sendeleistung
\item $S_{11} \leq$ 10 dB
\item Preis und Aufwand
\item ...
\item ...


\end{itemize}
\begin{tabular}{l|c|c|c|c}
\hline 
Nr. & Anforderung & Beschreibung & Wert & nicht erfüllt \\ 
\hline 
\hline 
001 & f & ISM Band  & 2.4-2.5 GHz & \\ 
\hline 
002 & f & Handgerät lxbxh & lxbxh &   \\ 
\hline 
003 & f & symmetische Speisung des Antennensystems &  \\ 
\hline 
004 & f &  &  \\ 
\hline 
• &  &  &  &  \\ 
\hline 
• &  &  &  &  \\ 
\hline 
• &  &  &  &  \\ 
\hline 
• &  &  &  &  \\
\hline 
• &  &  &  &  \\ 
\hline 
• &  &  &  &  \\ 
\hline 
• &  &  &  &  \\ 
\hline
• &  &  &  &  \\ 
\hline 
• &  &  &  &  \\
\hline 
\end{tabular} 
\subsection{Ziele der Arbeit}
\todo{formuliere SMART Ziele}
Es soll ein Funktionsmuster für eine integrierte Loop Antenne für das ISM  2.4-2.5GHz Band hergestellt werden.\\

Auf der Basis der Theorie der elektrisch kleinen Antennen wird ein Entwurf für ein Antennensystem im 2.4 - 2.5 GHz Band designed. Der Entwurf wird simuliert und dokumentiert. \\
Ein Anpassnetzwerk muss für die komplexe Ausgangsimpedanz des Transsivers dimensioniert, simuliert, hergestellt und ausgemessen werden.\\
Der simulierte Entwurf des gesamten Antennensystems wird produziert und dient als Funktionsmuster. \\
Die Antenneparameter des Funktionmusters müssen gemessen und dokumentiert werden.\\ Abweichungen zwischen der Simulation und den Messresultaten sollen dokumentiert und bewertet werden.
\subsubsection{Meilensteine}
Es werden die folgenden vier Meilensteine definiert. Die Meilensteine
beinhalten   eine Sammlung von Aufgaben und haben einen
Fertigstellungstermin, sie markieren das Ende einer Projektphase. Sie
werden für das Projektcontrolling verwendet. Beim erreichen eines Meilensteins werden die Arbeiten bewertet und es werden Beschlüsse über den weiteren Projektverlauf gefällt.
\todo{fünf Meilensteine definieren}
	\begin{itemize}
		\item MS 1 Theorie und Recherchenphase abgeschlossen und zu 89\% dokumentiert, ein Anforderungsdokument wurde erstellt
		\item Zwischenpräsentation, vorstellen der ersten vier Antennenkonzepte
		\item iterativ wurde eine Antennsystem gefunden welches den Anforderungen entspricht
		\item Engeenieringmodel ist gefertigt und dokumentiert

	\end{itemize}
Es werden die folgenden vier Meilensteine definiert. Die Meilensteine beinhalten   eine Sammlung von Aufgaben und haben einen Fertigstellungsterminn. Sie werden für das Projektcontroling verwendet. Beim erreichen eines Meilensteins werden die Arbeiten bewertet und es werden Beschlüsse über den weiteren Projektverlauf gefällt.



