\section{Einleitung}
bla bla bla \cite{meinke1992taschenbuch} und aus
\cite{kark2006antennen} aber  blabla a\cite{} und natürlich auch
\cite{gustrau2011hochfrequenztechnik} vieles von
\cite{elliott2003antenna} und \cite{emant} sowie von \cite{harrington1961time}
\begin{itemize}
\item Weshalb die Arbeit
\item Vorstellung des Industriepartners Flytec
\item Was wird in dieser Arbeit dokumentiert
	\begin{itemize}
		\item Antennentheorie und Abstrahlverhalten
		\item Entwurf einer kompakten Antenne
		\item Simulation von kompakten Antenne
		\item Fertigung eines Funktionsmusters
		\item Messung von Antennenparameter
		\item Vergleich zwischen Theorie und Praxis der Antennenparameter
		\item Bewertung des Funktionsmusters
	\end{itemize}
\end{itemize}
\subsection{Aufbau der Arbeit}
Übersicht des BDA Projekt welche Phasen es gibt.\\
\\
In welchem Kapitel was zu finden ist.\\
\begin{itemize}
\item Thematischer Überblick verschaffen
\item Grobplanung
\item Kickoff mit Industriepartner Flytec
\item Ziel definieren, Meilensteine definieren, Arbeitspakete bilden
\item Allgemeine Antennentheorie
\item Entwurf der Antenne
\item Simulation der Antenne
\item Funktionsmuster herstellen
\item Abstrahlcharakteristik messen
\item iterativ wiederholen bis Designvorgaben erreicht werden
\item Dokumentieren der Designrelevanter Theorie

\item Dokumentieren des Simulationsverfahrens
\item Dokumentieren des Messverfahrens
\item Vergleich zwischen Simulation und Messungen dokumentieren
\item Dokumentieren des Projektverlauf
\end{itemize}

