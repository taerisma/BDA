\newpage
\section{Einleitung}
Gleitschirmpiloten müssen sich während eines Fluges stets orientieren können, um wieder sicher auf dem Boden zu landen. Da die Orientierung in der Luft ohne fixe Orientierungspunkte sehr anspruchsvoll ist, greifen die meisten Piloten auf technische Hilfsmittel zurück. Beispielsweise kommen Variometer zur Anwendung, welche den Luftdruck messen und somit Höhenunterschiede dokumentieren. Der Pilot kann daraus schliessen, ob er sich im Steig- oder Sinkflug  und auf welcher Höhe er sich gerade befindet. Des Weiteren nutzen viele Piloten GPS-Geräte zur Positionsbestimmung. Um den Piloten unabhängig vom Hilfsgerät zu machen, erfolgt die Informationsvermittlung teilweise akustisch. So wird der Sinkflug  mit einem Piepton signalisiert, während beim Steigflug keine akustische Informationsübermittlung erfolgt. Die Firma Flytec stellt seit über 30 Jahren verschiedene Fluginstrumente für die Tuchfliegerei her. Ihre Instrumente können am Rumpf oder am Oberschenkel getragen werden und die Informationen werden auf einer berührungssensitiven Anzeige dem Piloten zur Verfügung gestellt.\\

%%%%%%
\colorbox{yellow}{\parbox[t]{\textwidth}{
Die meissten Gleitschirmpiloten greifen auf technische Hilfsmittel zurück. Beispielsweise kommen Variometer zur Anwendung, welche den Luftdruck messen und mit einem Piepton eine Information an den Piloten geben, ob er sich im Steig- oder Sinkflug befindet. Weiter werden Barometer da zu verwendet um den relative Höhe auf Grund der Luftdruckänderung aufzeichnen und dem Pilote visuell zur Verfügung stellen. Des Weiteren nutzen viele Piloten GPS-Geräte um ihre Flugruten aufzuzeichnen. Aus diesen Daten kann ein Pilot schliessen wann und wo bei welchen Wetterlagen er eine gute Thermik vorfand.\\ 
Die Firma Flytec stellt seit über 30 Jahren verschiedene Fluginstrumente für die Tuchfliegerei her. Ihre Instrumente können am Rumpf oder am Oberschenkel getragen werden und die Informationen werden auf einer berührungssensitiven Anzeige dargestellt.\\
}}
%%%%%

In dieser Arbeit soll für die Fluginstrumentenserie \glqq Connect 1 \grqq eine Kompaktantenne entwickelt werden. Diese wird im Rahmen des \glqq near pilot network\grqq \ zur Anwendung kommen, mit dem Ziel in Zukunft die Geräte  über eine Bluetooth-Verbindung mit einem Smartphone zu koppeln. In den folgenden Abschnitten wird die Ausgangslage dokumentiert und das bisherige Antennensystem beschrieben. In einem weiteren Schritt wird die Theorie der Kompaktantennen erarbeitet, um das Abstrahlverhalten der verschiedenen Antennen besser zu verstehen und die anschliessenden Simulationen sowie die Antennenmessungen fundiert interpretieren zu können. Aus einer Vorauswahl wird das vielversprechendste Konzept ausgewählt und für den Einsatz in einem Handgerät optimiert. Abschliessend soll ein Fazit gezogen und weitere Entwicklungsmöglichkeiten vorgeschlagen werden.

