\newpage
\section{Einleitung}
Gleitschirmpiloten müssen sich während eines Fluges stets orientieren können, um wieder sicher auf dem Boden zu landen. Da die Orientierung in der Luft ohne fixe Orientierungspunkte sehr anspruchsvoll ist, greifen die meisten Piloten auf technische Hilfsmittel zurück. Beispielsweise kommen Variometer zur Anwendung welche den Luftdruck messen und somit Höhenunterschiede dokumentieren. Der Pilot kann daraus schliessen, ob er sich im Steig- oder Sinkflug befindet und auf welcher Höhe er sich gerade befindet. Des Weiteren nutzen viele Piloten GPS-Geräte zur Positionsbestimmung. Um den Piloten unabhängig vom Hilfsgerät zu machen, erfolgt die Informationsvermittlung teilweise akustisch. So wird der Sinkflug zum Beispiel mit einem Pipton signalisiert, während beim Steigflug keine akustische Informationsübermittlung erfolgt. Die Firma Flytec stellt seit über 30 Jahren verschiedene Fluginstrumente für die Tuchfliegerei her. Die Instrumente können am Rumpf oder am Oberschenkel getragen und positioniert werden. Die Informationen werden auf einer berührungssensitiven Anzeige dem Piloten zur Verfügung gestellt. 

In dieser Arbeit soll für die Fluginstrumentenserie „Connect 1“ eine Kompaktantenne entwickelt werden. Diese wird im Rahmen des \glqq near pilot network\grqq \ zur Anwendung kommen. Mit dem Ziel, in Zukunft die Geräte der „Connect 1“ Serie über eine Bluetooth-Verbindung mit einem Smartphone zu koppeln. Initial wird die Ausgangslage dokumentiert und das bisherige Antennensystem beschrieben. In einem weiteren Schritt wird die Theorie der Kompaktantennen erarbeitet. Dies hilft das Abstrahlverhalten besser zu verstehen und die anschliessenden Simulationen sowie die Antennenmessungen zu interpretieren. Aus einem Vorprojekt wird das vielversprechendste Konzept ausgewählt und für den Einsatz in die Geräte Serie „Connect 1“ der Firma Flytec AG optimiert. Die Simulationen wiederum werden mit der Theorie verglichen. Abschliessend soll ein Fazit gezogen und weitere Entwicklungsmöglichkeiten vorgeschlagen werden.

