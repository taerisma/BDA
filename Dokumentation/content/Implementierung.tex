\newpage 
\thispagestyle{empty}
\chapter{Implementierung}\label{sec:Implementierung}
Wie im Kapitel \ref{sec:Interpretation_Dipol} erwähnt ist, wird die Antennenstruktur weder auf einer FR4 Kunstharzplatte noch auf eine Plastikfolie aufgedruckt. Da einerseits die Produktionskosten von Kleinserien im Verhältnis zum Nutzen eher klein sind, zum andern benötigt jeder Fertigungsdurchgang von gedruckten Antennen  mindestens ein bis drei Tage. Daher werden die zu Testenden Antennen aus Kupferfolie geschnitten. So können Änderungen des Design sehr schnell umgesetzt werden.\\

Aus der Interpretation der Dipolantennen Simulationen \ref{sec:Interpretation_Dipol} ist hervorgegangen, dass das Abstrahlverhalten eine Dipolantenne mit der breite von 1 mm und der Dicke von 26$\mu$ und einer Länge von xx mm ausgemessen wird.\\

Die Antenne wird mit dem Slapell aus dem Kupferklebeband ausgeschnitten. Das ist in der Abbildung Bliii gezeigt.\\
%\begin{figure}[!ht]
%	\centering
%	\includegraphics[width=4cm]{content/bilder/3D_.pdf}%
%	\caption{3D }
%	\label{fig:3D}
%\end{figure}

In der Abbildung bluuu zeigt die Antenne mit eimem 10 cm langen Koaxialkabel.\\
%\begin{figure}[!ht]
%	\centering
%	\includegraphics[width=4cm]{content/bilder/3D_.pdf}%
%	\caption{3D }
%	\label{fig:3D}
%\end{figure}

Die Abbidlung Bla zeigt wie die Antenne im Fluginstumet positioniert ist.\\
%\begin{figure}[!ht]
%	\centering
%	\includegraphics[width=4cm]{content/bilder/3D_.pdf}%
%	\caption{3D }
%	\label{fig:3D}
%\end{figure}


Aus den Simulationen ist der erwartete |$S_{11}$| Verlauf im Freiraum und im Gerät bekannt.\\

Der gemessene $S_{11}$ Verlauf im Freiraum wird dem simulierten $S_{11}$ im Freiraum der Dipolantenne in Abbildung xx gegenübergestellt.\\
%\begin{figure}[!ht]
%	\centering
%	\begingroup
%	\inputencoding{latin1}
%	\input{content/bilder/Messung/Messung_Sim_S11_Freiraum.tikz}
%	\endgroup
%	\caption{Vergleich des gemessen und simulierten $S_{11}$ Werts im Freiraum}	\label{S11_Messung_Simulation_Dipolantenn_Freiraum}
%\end{figure}

Der gemessene $S_{11}$ Verlauf im Gerät wird dem simulierten $S_{11}$ im Gerät der Dipolantenne in Abbildung xx2 gegenübergestellt.\\
%\begin{figure}[!ht]
%	\centering
%	\begingroup
%	\inputencoding{latin1}
%	\input{content/bilder/Messung/Messung_Sim_S11_Geraet.tikz}
%	\endgroup
%	\caption{Vergleich des gemessen und simulierten $S_{11}$ Werts im Geraet}	\label{S11_Messung_Simulation_Dipolantenn_Freiraum}
%\end{figure}

Eine qualitatives Bild des Abstrahlverhaltens der Dipolantenne im Gerät zeigt die Abbildung xx3.\\
%%%%%%%%%%%%%%%%%%%%%%%%%%%%%%%%%%%%%%%%%%%%%%%%%%%%%%%%%%%%%%%%%%%
%\begin{figure}[!ht]
%	\centering
%	\includegraphics[width=4cm]{content/bilder/3D_.pdf}%
%	\caption{3D }
%	\label{fig:3D}
%\end{figure}
%%%%%%%%%%%%%%%%%%%%%%%%%%%%%%%%%%%%%%%%%%%%%%%%%%%%%%%%%%%%%%%%%%%
Ein Schnitt durch die xy-Ebende bei z=0 ist in Abbildung xx4 dargestellt.\\
%\begin{figure}[!ht]
%	\centering
%	\begingroup
%	\inputencoding{latin1}
%	\input{content/bilder/Messung/xyGeraet.tikz}
%	\endgroup
%	\caption{xy-Ebene}\label{fig:xy_gemessen}
%\end{figure}
Ein Schnitt in der xz-Ebene bei y=0 ist in der Abbildung xx5 abgebildet.\\
%\begin{figure}[!ht]
%	\centering
%	\begingroup
%	\inputencoding{latin1}
%	\input{content/bilder/Messung/xzGeraet.tikz}
%	\endgroup
%	\caption{xy-Ebene}\label{fig:xz_gemessen}
%\end{figure}
Die Effizienz der Dipolantenne ist in Abhänigkeit der Frequenz in der Abbildung xx6 gezeigt\\
%\begin{figure}[!ht]
%	\centering
%	\begingroup
%	\inputencoding{latin1}
%	\input{content/bilder/Messung/Effizienz_Geraet.tikz}
%	\endgroup
%	\caption{xy-Ebene}\label{fig:Effizienz_gemessen}
%\end{figure}