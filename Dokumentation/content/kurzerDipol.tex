\newpage
\subsubsection{Der kurze Dipol }\label{sec:kurzerDipol}
Der kurze Dipol kann wie der $\lambda /2$ Dipol als eine Reihe von Elementaren Dipolen angesehen werden. Die einen schlanken Stromfaden bilden. Auch für den kurzen Dipol gilt, dass der Durchmesser d des Dipols viel klein ist als die Länge der Antennenstäbe. 
Die Terme $cos(kl*cos(\theta)) $ und $cos(kl)$ aus der Gewichtungsform Gleichung \ref{eq:Gewichtungsfunktion}
 des $\lambda/2$ Dipol können mit einer Reihe angenähert werden, sofern $kl$ klein ist.
\begin{equation}
a_{\theta}(\theta)=-kl^{2}I_{m}sin(\theta) \lbrack 1- \frac{(kl)^{2}}{12}(1+cos^{2}(\theta))+...\rbrack
\end{equation}
%Ellito 2.15 Seite 64
Der Eingangsstrom eines kurzen Dipols ist gegeben durch:
\begin{equation}
I=I_{m}sin(kl)=I_{m}\lbrack kl - \frac{(kl)^{3}}{3!} +... \rbrack
\end{equation}
%Ellito 2.16
 Für kleine Längen wie 2l = $\lambda/4 $ kann ohne grossen Fehler die Gewichtungsfunktion als 
%Ellito 2.17
\begin{equation}
a_{\theta}(\theta)=-kl^{2}I_{m}sin(\theta)=-Ilsin\theta
\end{equation}
angenommen werden.\\
Wie beim $\lambda/2$ Dipol findet man bei einem kurzen Dipol ein vertikal polarisiertes E Feld. Das Feld ist etwas breiter, \todo{Öffnungswinkel angeben} jedoch wie beim $\lambda/2$ Dipol doughnutförmig. Die Impedanz des kurzen Dipols ändert sich jedoch drastisch gegenüber dem $\lambda/2$ Dipol. Mit der Impedanz ist auch die winkelabhängige Leistungsdichte wie folgt gegeben:
%Ellito 2.18
\begin{equation}
P(\theta,\varphi)=\frac{(kl)^{2}\eta I^{2}}{2(4\pi r)^{2}}sin^{2}\theta
\end{equation}
Die abgestrahlte Leistung eines kurzen Dipols bei dem über die ganze Kugeloberfläche mit dem Radius $r$ integriert wurde, kann mit der Formel \ref{eq:Rrrad} berechnet werden.
%Ellito 2.19
\begin{equation}\label{eq:Rrrad}
R_{rad}=20 \left(\frac{\pi L}{\lambda} \right) ^{2}
\end{equation}



Die aus der abgestrahlten Leistung ergebende maximale Richtwirkung eines kurzen Dipols wird  mit einem isotopen Strahler verglichen. Dies ergibt nach Formel \ref{eq:Directivity} einen Wert für den Richtfaktor D von 1.5. Dieser Wert ist nicht viel weniger als bei einem $\lambda/2$ Dipol mit einem D Wert von 1.64. Das entspricht $10\log{1.5}=1.76dBi$.
Der Strahlungswiderstand kann mit der Umformung des $Prad=1/2 I^{2}Rrad$ umgestellt werden. \\
Man findet :
$Rrad=20\left(\piL/\lambda\right)^{2}$ \\
%Eliott
Dabei wird L als 2l und somit als länge der beiden Dipolarme angenommen.
Wenn ein Dipol sehr kurz wird, zum Beispiel $2l=\lambda/8$, dann wird $R_{rad} = 3 Ohm$. Dieser Wert ist merklich kleiner als   die 73 Ohm   eines $\lambda/2$ Dipols. Der Effekt auf den reaktiven Anteil der Eingangsimpedanz ist noch dramatischer. Für einen endlich dünnen Dipol mit der Dicke d, ist die Reaktanz der Eingangsimpedanz eines $2l=\lambda/2$ Dipols positiv. Die Reaktanz ist wenig unter Null, wenn die Dipollänge  $2l=\lambda/2$ entspricht. Wird der Dipol weiter gekürzt,  sinkt die Reaktanz sehr schnell ins Negative. Wenn  $2l=\lambda/8$ ist,  sind Werte für X grösser als 1000 Ohm kapazitiv keine Seltenheit. 


