
\chapter{Reflexion}
\section{Projektbezogene Reflexion}
Wie ist das Projekt verlaufen\\
Entscheidungsstellen\\

An dieser Stelle soll der Projektverlauf und einige Entscheidungsstellen aufgezeigt werden.\\
Zu Beginn der Arbeit wurde ein Projektplan erstellt. Dieser beinhaltete alle Meilensteine und die dazugehörenden Projektphasen. Für die Recherche und das Erlangen der Antennentheorie benötigte ich länger benötigt als geplant. Die Simulationen und das Erstellen der Funktionsmuster erfolge in der dafür geplanten Zeit, für die Auswertung und Dokumentation der Simulationsdaten musst mehr Zeit in Anspruch genommen werden, als ursprünglich geplant war.
Nach der Einarbeitung in die elektrisch kurzen Antennen, die Antenneparameter und die Phänomene der elektromagnetischen Wellenausbreitung konnte mit den ersten Simulationen begonnen werden. Nach einfachen Beispielen und Tutorials galt es die Antennenparameter von Dipol Antennen und Loop Antennen unter Berücksichtigung von ABS-Kunststoff im Nahfeld zu studieren.  Die ersten Erkenntnisse aus der Antennentheorie und den Simulationen von symmetrischen Antennen konnten an der Zwischenpräsentation aufgezeigt werden. Die Simulationen der symmetrischen Antennen haben gezeigt, dass die Dipol Antenne am vielversprechendsten für den Einsatz im „Connect 1“ Gerät zeigt.\\
Es wurden vier Varianten einer Dipolantenne für den Einsatz im Fluggerät optimiert, simuliert und Gefertigt. Untersucht und verglichen wurde ihr Resonanzverhalten, die Antennenimpedanz sowie ihre Abstrahleffizienz. 
Die Herstellung eines Funktionsmusters wurde Dokumentiert und anschliessend mit dem StarLab Antennenmessgerät die Antennenparameter aufgenommen. Die gemessenen Werte wurden mit denen aus der Simulation der Antenne im Gerät verglichen und interpretiert.
In einer Diskussion wurde ein Vergleich aus der Theorie mit den Simulationen des Designs und den effektiven Resultate gezogen. Eine Interpretation der erhalten Resultate gegebene. Ein Rückblick auf die Aufgabenstellung mit einem Vergleich, was gefordert war und was effektiv erreicht wurde zeigt den Stand der Bachelor Arbeit auf. Ein Empfehlung für die Flytec AG schliesst die Diskussion ab.  







\section{Persönliche Reflexion}
Persönliche Reflexion
Das Thema der Antennentechnik hat mich bereits zu Beginn der Arbeit sehr interessiert und fasziniert. Im Wissen, dass mir eine fundierte Theorie helfen wird, die Simulationsmodelle zu verstehen und die Antennenparameter richtig zu interpretieren, habe ich mich zu Beginn dieser Arbeit intensiv mit der Antennentheorie auseinandergesetzt. Es war für mich interessant zu sehen, wie nach dem Auffrischen der bereits bekannten Antennengrundlagen, der Module TKOM und EMNT klar wurde, welche Defizite und Lücken im Verständnis der Antennentheorie vorhanden war. Diese konnten durch Lesen und Erarbeiten von Bücher, Papers und Artikeln zum Thema Antennen und deren elektromagnetischen Felder geschlossen werden. 
Die Einarbeitung in das Simulationstool EMPIRE XPU habe ich anhand der vorhandenen Beispiele und der Tutorials gemacht. Die ersten Simulationsergebnisse waren nach kurzer Zeit erreicht, jedoch hatte ich einige Schwierigkeiten mit den Simulationseigenschaften der Software. Die Beispiele auf der Internetseite des Softwarebetreibers haben mir oft weitergeholfen und nach einigen Versuchen wusste ich die Simulationssoftware immer besser zu nutzen. Während dieser Arbeit konnte ich einige wichtige Erfahrungen im Umgang mit einem weiternen Simulationstool erlangen. Zu diesen gehören: Es ist sehr wichtig, vor jeder Simulation eine Erwartungshaltung festzuhalten, nur so können die erreichten Simulationsresultate vernünftig ausgewertet werden. Dafür ist das Verständnis der Theorie und Kenntnisse über das Verhalten des elektromagnetischen Feldes unter Einwirkung von unterschiedlichen Materialien im Nahfeld essenziell. Um lange Simulationszeiten zu vermeiden, ist es wichtig, die Vernetzung und Menge der Simulationspunkte bewusst zu wählen. Um bei komplexen Antennenstrukturen lange Simulationszeiten zu vermeiden, lohnt es sich, sich zu Beginn der Simulation mit dem Mesh und deren Einstellungen auseinander zu setzen. Durch die intensive Auseinandersetzung mit der Simulationssoftware, dem Interpretieren der erhaltenen  Simulationsresultate und das anschliessende diskutieren der Resultate mit meinem Betreuer Prof. Marcel Joss konnte ich mein Verständnis von elektromagnetischen Strahlern ausbauen. Es war für mich sehr befriedigend, wenn der Vergleich zwischen der Abstrahltheorie der Antennen und den Schnittbildern des Richtidagramms Parallelen gezogen werden konnten. Weiter habe ich viel gelernt beim Umgang mit Simulations- und Messdaten. Beim Exportieren der Simulationsdaten und anschliessenden zusammenführen der Messwerte mit MATLAB konnte ich meine MATLAB Kenntnisse auffrischen. Allgemein zu den Simulationen ist zu sagen, dass ich zu lange an unwesentlichen Details gearbeitet habe. Ich hätte schneller und effizienter mein Ziel erreicht, wenn ich nur einige wenige Simulationen zum groben Verhalten angestellt hätte und mich dann immer weiter vorgearbeitet hätte. Diese Erkenntnis ist für mich sehr wichtig.
Als die Designphase beendet wurde, waren die zu produzierenden Funktionsmuster bekannt. Diese herzustellen und anschliessend auszumessen hat mir viel Freude bereitet. Rückblickend habe ich zu früh zu viele Funktionsmuster produziert. Es wäre besser gewesen, nur zwei Funktionsmuster herzustellen, deren Antennenparameter aufzunehmen und zu dokumentieren. Im Endeffekt habe ich viel mehr Funktionsmuster hergestellt und ausgemessen als ich dokumentieren konnte.
Das Aufnehmen der Antennenparameter und das Vergleichen der Simulationsresultate war für mich sehr wertvoll. Die Vergleiche der qualitativen Feldverteilung waren erst richtig möglich, nachdem die Minimal- und Maximalwerte des Antennengewinns, der Simulationssoftware identisch eingestellt wurde, wie der Minimal- und Maximalwerte des StarLab. So konnte der Gewinn einer Antenne, anhand der Farbe des 3D Richtdiagramms ermittelt werden. Der Umgang mit dem StarLab Antennenmessgerät, war gewöhnungsbedürftig aber spannend. Es musste einige Zeit investiert werden, um herauszufinden, wie die Schnittbilder aus dem errechneten Fernfeld zu erstellen sind. Um diese zu erstellen, musste ich mich noch einmal mit den Kugelkoordinaten auseinandersetzen und die anschliessende Besprechung mit dem Betreuer  Prof. Marcel Joss war interessant und lehrreich. Nur gemeinsam und anhand bereits existierender Messdaten konnte ausfindig gemacht werden, wie die Schnitte in der xy-Ebene und in der xz-Ebene erstellt werden. Um die Schnittbilder der xy-Ebene und der xz-Ebene in MATLAB zu erstellen habe ich viel Zeit investiert. Das Ergebnis ist nur befriedigend. Hätten die Bilder keinen Offset und wäre das Grid in Polarkoordinaten, so wäre die richtungsabhängige Feldausbreitung viel einfacher abzulesen. Auch wenn ich das MATLAB Skript nicht abschliessen konnte, bin ich der Meinung, dass der Weg über ein MATLAB Skript für die Darstellung der Daten aus dem StarLab für zukünftige Arbeiten weiter zu verfolgen ist.  
Ich bin überzeugt, dass ich mit der Wahl dieser BDA ein sehr interessantes Projekt gewählt habe, welches bei der Lösungsfindung viele Freiheitsgrade zulies und ich viel  lernen konnte. Die Besprechungen mit meinem Betreuer Prof. Marcel Joss waren für mich lehrreich und haben mir Wege aufgezeigt, um Probleme und Schwierigkeiten zu umgehen oder zu meistern. Am meisten nehme ich für meine weitere Laufbahn die Erfahrungen aus dem Umgang mit dem Simulationstool und den anschliessenden Vergleichen aus den Messungen des Funktionsmusters mit.
Ich bin überzeugt, dass die vorliegende Arbeit der Firma Flytec AG eine Hilfe sein wird, die zukünftige Generation der Bluetooth Antenne im „Connect 1“ Gerät neu zu gestalten und somit den Funktionsumfang dieser Geräteserie zu erweitern.

