\newpage
\section*{Abstract}
Since over 30 years, the company Flytech AG is producing different aviation instruments for the sector paraglider and hang glider. The aviation instrument “Connect 1” includes a Bluetooth network, to allow a data connection with a smart phone. To use this Bluetooth connection efficiently, the antenna of the aviation instrument needs to be improved. The study “Entwurf einer Kompaktantenne” from Pascal Schantl (06. June 2014) demonstrates that the selection of the antenna as well as its positioning has a significant impact on the radiation and the related systems. The objective of this study is to find a functional model for the 2.4 GHz antenna that is technically realizable.\\
In the first step the technical foundation for different types of antennas are evaluated to match the required functional model. In the next step, the simulation phase, the loop and dipole antenna (both symmetrically fed) are tested through simulation in the tool Empire XPU.  In the development phase, different design variations for the installation into “Connect 1” will be tested for the antenna with the most promising result in terms of radiation.\\
During the simulation phase the dipole antenna proved to be the most suitable for the use in “Connect 1”. As such four different variations of dipole antennas were chosen to be further analysed in the development phase. The dipole antenna with a size of 3 mm and a length of 50.25 mm is closely fulfilling the requirement. With an antenna impedance of (30+j4) $\Omega$ the radiating efficiency results in 49$\%$ with a target frequency of 2.45GHz.\\
The dipole antenna with the specifications as described above achieves the defined antenna volume as well as the transmission bandwidth. Additionally, by using a symmetric antenna the balun is no can be removed. In this project the radiation efficiency of the functional model was strongly improved compared to the currently used bluetooth antenna in “Connect 1”, however, does not reach the results of the simulation. An additional improvement of the radiation efficiency could probably be achieved by an optimization of the antenna structure to guarantee a better connection to the transceivers. This study builds a foundation for further optimization of the “Connect 1” antenna.
