\documentclass[10pt,a4paper]{article}
\usepackage[utf8]{inputenc}
\usepackage[german]{babel}
\usepackage[T1]{fontenc}
\usepackage{amsmath}
\usepackage{amsfonts}
\usepackage{amssymb}
\usepackage{graphicx}
\usepackage[left=2cm,right=2cm,top=2cm,bottom=2cm]{geometry}

%\documentclass[10pt,a4paper]{article}
%\usepackage{layout}
\begin{document}
%\chapter{Gesprächsprotokoll}
%\maketitle
\section*{SW7}

\subsection*{Gesprächsthemen}

%Version:1 \\
Besprechung im B301.\\
Kurze Besprechung des erarbeiteten Antennen Grundlagen Teil.\\
Dieser ist teil der BDA Dokumentation und beinhaltet die Grundlagen der Abstrahlung von elektromagnetischen Wellen von Loop- und Dipol Antennen. Der Antennenteil ist gut, sollte aber noch um Grafiken erweitert werden, die das Abstrahlverhalten von Dipol- und Loop Antennen verdeutlicht. Weiter sollen für beide Antennentypen Grafiken erstellt werden, die Feldverteilung um die Antennen zeigen.\\
Weiter wurde die  Zwischenpräsentation besprochen.\\
Die Zwischenpräsentation, soll Auskunft über das bereits erreichte und die Ziele geben. Sie soll einen \glqq soll und ist \grqq Vergleich beinhalten. Die bereits erarbeitete Theorie der elektrisch Kurzen Antennen soll aufgezeigt werden. Erkenntnisse aus der Theorie und den gemachten Antennensimulationen  aufgezeigt werden. Ein Projektzwischenfazit soll den weitern Verlauf der BDA aufzeigen.\\
Simulationen des Abstrahlverhalten von Dipol- und Loop Antennen helfen ihre Charakteristik aufzuzeigen


\vspace{10 mm}
Für die BDA wichtige Punkte die aus dem Gespräch hervorgegangen sind:
\begin{itemize}
	\item Antennen Theorie ist gut, soll aber mit Grafiken veranschaulicht werden
	\item Simulationen von Loop und Dipol Antennen
	\item Simulation von Dipol und Loop Antenne im Freiraum
	\item Die mechanische Länge der Antennen variieren
	\item Simulation von Dipol und Loop Antenne  mit ABS Kunststoffstück

\end{itemize}

\subsection*{Wichtige Beschlüsse}
Inhalt des Antennen Grundladen Teils
\begin{itemize}
	\item Grafiken für den Antennen Grundlagen Teil erstellen
	\item Präsentation erstellen
 erinnern
\end{itemize}
\end{document}