\documentclass[10pt,a4paper]{report}
\usepackage[utf8]{inputenc}
\usepackage[german]{babel}
\usepackage[T1]{fontenc}
\usepackage{amsmath}
\usepackage{amsfonts}
\usepackage{amssymb}
\usepackage{graphicx}
\usepackage[left=2cm,right=2cm,top=2cm,bottom=2cm]{geometry}

%\documentclass[10pt,a4paper]{article}
%\usepackage{layout}
\begin{document}
\chapter{Gesprächsprotokoll}
%\maketitle
\section{SW2}
\subsection{Gesprächsthemen}

%Version:1 \\

Das Hauptthema war die Produktbesprechung mit dem Industriepartner der
Flytec vertreten durch Herr Erich Lerch. Für das Gerät ist ein
Antennensystem im 2.4 GHz ISM Band zu design. Das System wird von einem Bluetooth low Energie Chip von Texas Instruments angeregt.
Für die BDA wichtige Punkte die aus dem Gespräch hervorgegangen sind:
\begin{itemize}
	\item das Gehäuse ist aus ABS Kuststoff
	\item Für das Linkbuget soll ein Wirkradius von 10 im Freiraum angenommen werden
	\item Richtcharakteristik ist keine gefordert, wir definieren isotroph 
	\item es wird eine Reserve von 6 dB eingeplant
	\item als Empfangsgewinn wird 1 definiert
	\item $S_{11} \leq$ 10 dB
	\item Texas Instruments CC2541
	\item auf einen Balun verzichten ist der Wunsch
	\item symmetrisch angesteuerte Antenne
	\item es wird sich um ein linear polarisiertes System handeln
\end{itemize}
Das G der Sendeantenne ist zu finden. Der Frequenzbereich des Low Energie Bluetooth von 2.4 GHz bis 2.45 GHz ist zu prüfen.
\subsection{Wichtige Beschlüsse}

\begin{itemize}
	\item Herr Joss richtet die Dropbox ein
	\item Frequenzbereich des Low Energie Bluetooth prüfen
	\item Gewichtungsfaktor der Loop Antenne und des Dipols studieren
	\item wie verhält sich das $Z_{ant} $ bei verkürzter Antenne
	\item falls möglich die charakteristischen Eigenschaften von ABS prüfen
\end{itemize}
In einer späteren Phase des Projekt ist zu prüfen, ob und wie es
möglich ist, dass die komplexe Ausgangsimpedanz der Quelle durch
die Dimensionierung der Antennenleitung von der Quelle zum Fusspunkt
der Antenne eine Anpassung auf Z=50+j*0 Ohm vornimmt. Das spart einen
Balun ein und macht die Simulation einer Antenne erheblich einfacher,
denn es kann von einer Quelle mit  $Z_{aus} $ von reel 50 Ohm
ausgegangen werden.
\subsection{Fragen auf SW3}
\end{document}