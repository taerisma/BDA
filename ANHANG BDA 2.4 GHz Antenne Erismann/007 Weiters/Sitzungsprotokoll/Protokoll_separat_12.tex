
\documentclass[10pt,a4paper]{article}
\usepackage[utf8]{inputenc}
\usepackage[german]{babel}
\usepackage[T1]{fontenc}
\usepackage{amsmath}
\usepackage{amsfonts}
\usepackage{amssymb}
\usepackage{graphicx}
\usepackage[left=2cm,right=2cm,top=2cm,bottom=2cm]{geometry}

%\documentclass[10pt,a4paper]{article}
%\usepackage{layout}
\begin{document}
%\chapter{Gesprächsprotokoll}
%\maketitle
\section*{SW12}

\subsection*{Gesprächsthemen}

%Version:1 \\
\textbf{KW 48}\\
Kurze Besprechung 3 Woche nach der Zwischenpräsentation.\\


In der letzten Woche wurde eine Antennen evaluation gemacht. Es wurden die vor und Nachteile von Loop und Dipolentennen im Einsatz in der "Connect 1" Serie analysiert. Der Entscheid wurde aufgrund von Simulatonen und Theoretischen Charaktereigenschaften im Freiraum gefällt. \\
Der Entscheid für die weiteren Simulationen und die Fertigung von Antennen ist auf eine Dipolantenne gefallen.\\
Es wurde über die Eigenschaften der

\vspace{10 mm}
Für die BDA wichtige Punkte die aus dem Gespräch hervorgegangen sind:
\begin{itemize}
	\item Die bestehenden Bluetooth und WiFi Antenne auf die tatsächlichen Masse im Simulationsmodel kürzen
\item Eine im Freiraum getestet Kupferband Dipolantenne in das Simulationmodel an der vorgesehen Stelle platzieren
\item Antennen Parameter der im Simulationsmodel platzierten Dipolantenne prüfen
\item Abweichungen zwischen den Freiraumabstrahlung und der im Gerät verbauten Antenne dokumentieren
\item Antennendesign anpassen um die optimale Antennen Abstrahlcharakteristik an einer 50 Ohm Quelle zu erreichen
		
\end{itemize}

\subsection*{Wichtige Beschlüsse}
KW48: zwei bis drei Antennendesign im EMPIRE XPU \\
KW49: Antennen mit semi rigid Koaxialkabel versehen und Impedanz der Antennen bestimmen\\
KW49: Messen der Antennen im StarLab\\
KW50: Besprechen der Antennenmessungen\\
KW51: BDA Doku zu 80\% fertig\\


Weitere Arbeiten:
\begin{itemize}
\item Überlegungen zu den Messungen und den Simulationen betreffende der Koordinaten System
	\item Netzwerkanalystor die Referenzebene Anpassung (reference plane oder decomposition of port)
	\item Überlegungen zur Kabelführung der Speiseleitung bei den Starlabmessungen erstellen
\end{itemize}
\subsection*{Fragen}
\begin{itemize}
	\item ...
\end{itemize}
\end{document}

