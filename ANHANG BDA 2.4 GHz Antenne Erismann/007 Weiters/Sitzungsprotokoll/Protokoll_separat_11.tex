\documentclass[10pt,a4paper]{article}
\usepackage[utf8]{inputenc}
\usepackage[german]{babel}
\usepackage[T1]{fontenc}
\usepackage{amsmath}
\usepackage{amsfonts}
\usepackage{amssymb}
\usepackage{graphicx}
\usepackage[left=2cm,right=2cm,top=2cm,bottom=2cm]{geometry}

%\documentclass[10pt,a4paper]{article}
%\usepackage{layout}
\begin{document}
%\chapter{Gesprächsprotokoll}
%\maketitle
\section*{SW11}

\subsection*{Gesprächsthemen}

%Version:1 \\
\textbf{KW47}\\
Kurze Besprechung 2 Woche nach der Zwischenpräsentation.\\


Besprechung der Feldausbreitung von Loop- und Dipolantennen. Der Fokus liegt auf der Feldausbreitung im Kugelkoordinatensytem. Es wurde die E Feldkomonenten und die H Feldkomoponeten der symetrischen Antennen besprochen. Da die bestehende Bluetooth und die WiFi Antenne in der Hardware bestehen bleibt muss die Kopplung zwischen den Antennensystem unbedingt berücksichtig werden. Die Kopplung der Antennensysteme und die gegenseitige Beeinflussung wurde besprochenen. 

\vspace{10 mm}
Für die BDA wichtige Punkte die aus dem Gespräch hervorgegangen sind:
\begin{itemize}
	\item Weder die Dipol- noch die Loopantenne weisen in der Theorie an der vorgesehenen Stelle eine starke Kopplung mit den übrigen Antennensystemen auf
	\item Es werden 2 bis 3 Antennen Designs für die Fertigung erstellt
	
\end{itemize}

\subsection*{Wichtige Beschlüsse}
Entscheidung auf welchen Antennentyp sich die Prototypingphase konzentrieren soll aufgrund der folgenden Parameter gefällt werden:
\begin{itemize}
	\item $Z_{ant}$, $R_{rad}$,Richtwirkung, Abstrahleffizienz, Detuning durch das Umgebungsmaterial, Simulations- und Produktionsaufwand
	\item Entscheid der Loop oder Dipolantenne bis 19.11.2015 per Mail M. Joss
	\item bis Ende KW 48 liegen die Designs für die Fertigung bereit
\end{itemize}
\subsection*{Fragen}
\begin{itemize}
	\item ...
\end{itemize}
\end{document}


