%\documentclass[10pt,a4paper]{report}
%\usepackage[utf8]{inputenc}
%\usepackage[german]{babel}
%\usepackage[T1]{fontenc}
%\usepackage{amsmath}
%\usepackage{amsfonts}
%\usepackage{amssymb}
%\usepackage{graphicx}
%\usepackage[left=2cm,right=2cm,top=2cm,bottom=2cm]{geometry}

%\documentclass[10pt,a4paper]{article}
%\usepackage{layout}
%\begin{document}
\chapter{Gesprächsprotokoll}
%\maketitle
\section{SW1}
\subsection{Gesptächsthemen}
Das Hauptthema war die Grobplanung der BDA. Die zur verfügungstehend 15 Wochen werden in vier Phasen eingeteilt. Die vier Phasen sind.
\begin{itemize}
	\item Recherche- und Theoriephase
	\item Designphase
	\item Prototyping 
	\item Dokumentation des Engeneeringmodel
\end{itemize}
Die vier Phasen weden je mit einem Meilenstein enden. Die  die Meilensteine dienen gleichzeitig als Projektcontroling. Bevor in eine nächste Phase übergeganen wird, muss der Projektstand mit den Forderungen des Meilensteins überprüft werden. Zu diesen Zeitpunkten soll  ein Beschluss über den weiten Verlauf des Projekt gefällt werden.
\subsection{Wichitige Beschlüsse}
Die Recherche- und Theoriephase soll die folgenden Punkte beinhalten. 
\begin{itemize}
	\item Studium der beiden Grundantennen Monopol und Dipol
	\item Studium der beiden symetrischgespiesenen Antennen Dipol und Loop Antenne
	\begin{itemize}
		\item das elektrische Feld des Monopol
		\item das magnetische Feld des Monopol
		\item das elektrische Feld des Dipol
		\item das magnetische Feld des Dipol
		\item das elektrische Feld der Loop Antenne
		\item das magnetische Feld der Loop Antenne
		\item das Nah- und Fernfeld der Antennen
	\end{itemize}
		\item die matematische Beschreibung der Felder der elementaren symetrischen Antennen
		\begin{itemize}
		\item Herzscher Dipol
		\item Fitzgeradscher Dipol
		\end{itemize}
		\item Antennenipedanz des Dipols bei der Länge  $L=\lambda / 2$ und Monopol bei $L=\lambda / 4$.
		\item Wie ändern sich die folgenden Antenneparamter wenn die Länge der Antenne kürzer wird
	\begin{itemize}
		\item Impedanz der Antennen
		\item el. mag. Feld
		\item Strahlungswiderstand
		\item Bandbreite
	\end{itemize}
\end{itemize}
Die Theorie zu den Aufgeführten Punkten hilft, den Designprozess voranzutreiben. Zudem ist das Wissen über das elektromagnetische Verhalten der  Antennen nötig um die Simulationsergebnisse zu bewerten. Am Ende der Recherche- und Theoriephase sind die wesentlichen Erkenntisse zur Antennentheorie zu 80\% im Bericht dokumentiert. Weiterer bestandteil des Meilenstein ist das erstellen  Anforderungsdokument, welches die technischen Eigenschaften des Antennsystems beschreibt.\\
\\Bestandteil der  Design Phase sind:
\begin{itemize}
	 \item Vertraut werden mit dem Desing und Simulationstool Empire  Xccel
	 \item Erstellen eines abstrahierten Simulationsmodel  
	 \item Benennen und quantisieren der Simulationsfehler
	 \item erstellen von drei bis vier möglichen Antennenkonzepten
	 \item Simulation der Antennenkonzepte
	 \item Diskussion der Antennenkonzepte beinhaltet folgende Punkte
		 \begin{itemize}
		 \item Antennengüte Q
		 \item Impedanz
		 \item Strahlungswiderastant
		 \item Abstrahleffizienz
		 \item Richtcharakteristik
		 \item relative Bandbreite
		 \end{itemize}
 \end{itemize} 
Die Erkenntnisse aus den ersten Simulationen der Antennenkonzepte werden am Ende der Design Phase an der Zwischenpräsentation gezeigt. Auch soll auf das Verhalten von eletisch kurzen Antennen eingegangen werden und verschiede Einflüsse der Abstrahlcharakteristik und ihre Zusammenhänge sollen gezeigt werden. \\
Die Prototyping Phase ist von einem iterativen Prototyping Prozess getrieben. In dieser Prototyping Phase folgt das Simulieren auf  eine Designänderung und bei vielversprechenen Simulationen wird ein Funktionsmuster produziert und ausgemessen. Wenn die Messresultate nicht mit den Erwartungen der Simulation übereinstimmen, wird eine neue  Iteration gestartet, die widerum die Schritte: Design, Simulation, Prototype Fertigung, Messen und beurteilen beinhaltet.
   Die fünf Schritte einer jeden Iteration
   \begin{itemize}
	   \item Design
	   \item Simulation
	   \item Prototype ferigen
	   \item Messen
	   \item Beurteilen
   \end{itemize}

Die Abschliessende vierte Phase ist die Vorserienphase. Es gillt sämtliche Überlegungen und Erkenntisse aus den ersten drei Phasen zu dokumentiern und ein Engeneeringmodel der bis dahin  bestmöglichen Konfiguration des Antennensystems zu erstellen. 
\subsection{To do auf SW2}
\begin{itemize}
	\item Antennenparameter und Abstrahlung des el. mag. Feld der symetrischen Antennen recherchienen und dokumentiern.
	\item relevante Paramter für die Anforderungliste auflisten
\end{itemize}
\subsection{Fragen auf SW2}
%\end{document}