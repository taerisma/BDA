\documentclass[10pt,a4paper]{article}
\usepackage[utf8]{inputenc}
\usepackage[german]{babel}
\usepackage[T1]{fontenc}
\usepackage{amsmath}
\usepackage{amsfonts}
\usepackage{amssymb}
\usepackage{graphicx}
\usepackage[left=2cm,right=2cm,top=2cm,bottom=2cm]{geometry}

%\documentclass[10pt,a4paper]{article}
%\usepackage{layout}
\begin{document}
%\chapter{Gesprächsprotokoll}
%\maketitle
\section*{SW6}

\subsection*{Gesprächsthemen}

%Version:1 \\
Einleitung und Antennen Grundlagen wurden beschrieben. Trotz der Besprechung der Erwartungen wurde festgestellt, dass ich undter den Grundlagen  zuviel verstanden habe. Die Beschreibungder Grundlagen haben ich zuweit ausgelegt und daher keinen abschliessenden Bogen über die beschrieben Grundlagen ziehen können. Ich habe mich zuweit verstreut und daher nicht die gewünschte tiefe der Themen erreicht.\\ Nach einer Besprechung des geleitteten wird mir klar was  gewünscht ist.\\
Ich bin froh um die in der Besprchung  erhalten Informatioen.
Es hat mir gezeit, dass in der Vergangenheit oft über den Inhalt der Doku gesprochen wurde, aber ich einfach nicht das gleiche unter den Aussagen verstanden habe wie mein Betreuer. In Worten und im Gespräch war mir immer klar, was zu tun war, aber als ich an der Bearbeitung der Themen für den Grundlagenteil war, da war ich mir  nicht meher sicher und wollte viel zu viele Themen abdecken.  \\
Der Aufbau der Dokumnetation wurde Besprochen. Einleitung und Ausgangslage wurden einige Punkte genauer erläutert. Der Inhalt von Kapitel 2 "Antennen Grundlageng" wurde besprochen und ist in den \textbf{ Wichtigen Besüssen} festgehlten. \\
Für die Zwischenpäsentation sollen im Freiraum die Loop Antennne und die Dipol Antenne simuliert werden. \\
nach Möglichkeit sollen die Simulationen um ein kleines etwa 2.5mm dickes ABS Kunsstoffstück erweritert werden.\\
Für die nächste Besprächung soll ein erstes fazit aus den Erwartungegn der Theorie und den Simulationen besprochen werden.

\vspace{10 mm}
Für die BDA wichtige Punkte die aus dem Gespräch hervorgegangen sind:
\begin{itemize}
	\item Dokumentation Einleitung schreiben
	\item Dokumentation Aufgabenstellung schreiben
	\item Dokumentation Antennen Grundlagen schreiben
	\item Simulation von Dipol und Loop Antenne im Freiraum
	\item Simulation von Dipol und Loop Antenne  mit ABS Kunsstoffstück

\end{itemize}

\subsection*{Wichtige Beschlüsse}
Inhalt des Antennen Grundladen Teil
\begin{itemize}
	\item Dipolantenne
	\item Loop Antenne
	\item Abstrahlverahlten von Dipol und Loop Antenne
	\item $R_{rad}$ der Dipol und Loop Antenne
	\item 2D Feldverteilung von Dipol und Loop Antenne
\end{itemize}

\subsection*{Fragen auf SW7}
\end{document}