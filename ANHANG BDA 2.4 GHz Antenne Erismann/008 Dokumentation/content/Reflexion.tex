
\chapter{Reflexion}
\section{Projektbezogene Reflexion}
An dieser Stelle soll retrospektiv der Projektverlauf beurteilt und einige wichtige Entscheidungsstellen aufgezeigt werden. Zu Beginn der Arbeit wurde ein Projektplan erstellt. Dieser beinhaltete alle Meilensteine und die dazugehörenden Projektphasen. In die initiale Recherche- und Theoriephase investierte ich länger als geplant, um eine fundierte Wissensbasis in der Antennentheorie zu erlangen. Dies bewährte sich in der Simulation- und Designphase, welche in der dafür geplanten Zeit zielführend war. Das Erstellen der Funktionsmuster sowie die Auswertung und Dokumentation der Simulationsdaten in der Endphase „Funktionsmuster und Verifikation“ nahm wiederum mehr Zeit in Anspruch, als ursprünglich geplant war. Da bereits zu Beginn des Projekts ausreichend Reserven eingeplant wurden, konnte die Arbeit dennoch ohne Zeitdruck fristgerecht erstellt werden.\\
Nach Einarbeitung in die elektrisch kurzen Antennen, die Antennenparameter und die Phänomene der elektromagnetischen Wellenausbreitung konnte ich mit ersten Simulationen beginnen. Nach einfachen Simulationsbeispielen und Tutorials galt es die Antennenparameter von Dipol Antennen und Loop Antennen unter Berücksichtigung von ABS-Kunststoff im Nahfeld zu studieren. Die ersten Erkenntnisse aus der Antennentheorie und den Simulationen von symmetrischen Antennen konnte ich an der Zwischenpräsentation, dem zweiten Meilenstein der Arbeit, aufzeigen. Die Dipol Antenne war aufgrund der Simulationen am Vielversprechendsten für den Einsatz im „Connect 1“ Gerät.\\
Es wurden daher vier Varianten einer Dipol Antenne für den Einsatz im Fluggerät optimiert, simuliert, gefertigt und ausgemessen. Untersucht und verglichen wurden ihr Resonanzverhalten, die Antennenimpedanz sowie ihre Abstrahleffizienz. Aufgrund dieser Aspekte sowie dem Gesichtspunkt des Produktionsaufwandes wählte ich ein Funktionsmuster aus, dessen  Herstellung ich dokumentierte. Die anschliessenden Messungen erfolgten mit dem Antennenmessgerät StarLab, welches eine nicht zu unterschätzende Einarbeitungszeit erforderte. \\
In einer Diskussion zog ich einen Vergleich zwischen der Theorie und den Simulationen des Designs sowie den effektiven Messresultaten. Dies erlaubte mir einen Rückblick auf die Aufgabenstellung und den Vergleich zwischen den Anforderungen und den erreichten Zielen dieser Bachelor Arbeit. Die Interpretation der erhalten Resultate erlaubte mir letztendlich, eine Empfehlung für die Flytec AG geben zu können.  

\section{Persönliche Reflexion}
Das Thema der Antennentechnik interessierte und faszinierte mich bereits vor Beginn dieser Arbeit, weshalb ich mich letztendlich auch für diese Bachelor Diplom Arbeit entschieden habe. Im Wissen, dass mir eine fundierte Theorie helfen wird, die Simulationsmodelle besser zu verstehen und die Antennenparameter richtig zu interpretieren, habe ich mich zu Beginn dieser Arbeit intensiv mit der Antennentheorie auseinandergesetzt. Es war für mich interessant zu sehen, wie nach dem Auffrischen der bereits bekannten Antennengrundlagen der Module TKOM und EMNT klar wurde, welche Defizite und Lücken im Verständnis der Antennentheorie noch vorhanden waren. Diese konnten durch Lesen und Erarbeiten von Büchern, Papers und Artikeln zum Thema Antennen und deren elektromagnetischen Felder geschlossen werden. \\ 
Die Einarbeitung in das Simulationstool EMPIRE XPU habe ich anhand der vorhandenen Beispiele und der Tutorials gemacht. Die ersten Simulationsergebnisse waren nach kurzer Zeit erreicht, jedoch hatte ich anfangs einige Schwierigkeiten mit den Simulationseigenschaften der Software. Die Beispiele auf der Internetseite des Softwarebetreibers haben mir oft weitergeholfen und nach einigen Versuchen wusste ich die Simulationssoftware immer besser zu nutzen. Während dieser Arbeit konnte ich einige wichtige Erfahrungen im Umgang mit einem Simulationstool erlangen. Unter anderem ist es sehr wichtig vor jeder Simulation eine Erwartungshaltung festzuhalten, nur so können die erreichten Simulationsresultate vernünftig ausgewertet werden. Hierfür sind das Verständnis der Theorie und Kenntnisse über das Verhalten des elektromagnetischen Feldes unter Einwirkung von unterschiedlichen Materialien im Nahfeld jedoch essenziell. Um lange Simulationszeiten zu vermeiden ist es zudem wichtig, die Vernetzung und Menge der Simulationspunkte bewusst zu wählen. Bei komplexen Antennenstrukturen ist es besonders lohnenswert sich zu Beginn der Simulation mit dem Mesh und dessen Einstellungen auseinander zu setzen. \\
Durch die intensive Auseinandersetzung mit der Simulationssoftware, der selbständigen Interpretation der erhaltenen Resultate und der anschliessenden Diskussion der Resultate mit meinem Betreuer Prof. Marcel Joss konnte ich mein Verständnis von Antennen ausbauen. Es war für mich sehr befriedigend, wenn aus dem Vergleich zwischen der Abstrahltheorie der Antennen und den Schnittbildern des Richtidagramms Parallelen gezogen werden konnten. Weiter habe ich viel gelernt im Umgang mit Simulations- und Messdaten, so konnte ich beim Exportieren der Simulationsdaten und beim anschliessenden Zusammenführen der Messwerte im MATLAB meine MATLAB-Kenntnisse auffrischen. Allgemein zur Simulationsphase ist zu sagen, dass ich zu lange an unwesentlichen Details gearbeitet habe. Ich hätte schneller und effizienter mein Ziel erreicht, wenn ich nur einige wenige Simulationen zum groben Verhalten symmetrischer Antennen angestellt und mich dann immer weiter vorgearbeitet hätte. Diese Erkenntnis ist für mich sehr wichtig in Bezug auf diverse Arten von Entwicklungsprozessen.\\
Die in der Designphase ausgewählten Funktionsmuster herzustellen und anschliessend auszumessen, hat mir viel Freude bereitet. Rückblickend habe ich jedoch zu früh zu viele Funktionsmuster produziert. Es wäre besser gewesen, nur zwei Funktionsmuster herzustellen, deren Antennenparameter aufzunehmen und zu dokumentieren. Im Endeffekt habe ich viel mehr Funktionsmuster hergestellt und ausgemessen, als ich dokumentieren konnte.
Das Messen der Antennenparameter und das Vergleichen mit den Simulationsresultaten waren für mich sehr wertvoll. Dabei war für mich vor allem die Erfahrung von Bedeutung, dass bei der Beurteilung einer Graphik stets die dazugehörenden Achsenbeschriftungen mit ihren Skalen berücksichtigt werden müssen. So sind Vergleiche der qualitativen Feldverteilung beispielsweise erst vollumfassend möglich nach Anpassung der Minimal- und Maximalwerte des Antennengewinns in der Simulationssoftware an die Minimal- und Maximalwerte des Antennengewinns im Antennenmessgerät StarLab. Hiermit kann der Gewinn einer Antenne anhand der Farbe im 3D-Richtdiagramms ermittelt werden. Erfolgt diese Anpassung nicht, so kann lediglich die Form des 3D-Richtdiagramms, jedoch nicht die Feldstärke im Raum verglichen werden. \\
Der Umgang mit dem StarLab Antennenmessgerät war gewöhnungsbedürftig, aber spannend. Ich musste einige Zeit investieren, um herauszufinden wie die Schnittbilder aus dem errechneten Fernfeld zu erstellen sind. Hierfür setzte ich mich intensiv mit dem Kugelkoordinatensystem auseinander. Die anschliessende Besprechung mit meinem Betreuer Prof. Marcel Joss war interessant und lehrreich. Nur gemeinsam und anhand bereits existierender Messdaten konnte ausfindig gemacht werden, wie die Schnitte des 3D-Richtdiagramms in der xy-Ebene und in der xz-Ebene vom Antennenmessgerät erstellt werden. Um die Schnittbilder der xy-Ebene und der xz-Ebene in MATLAB zu erstellen habe ich erneut viel Zeit in ein MATLAB-Skript investiert. Das Ergebnis ist meiner Meinung nach leider nur befriedigend. In Schnittbilder ohne Offset und einem mit einem Grid in Polarkoordinaten wäre die richtungsabhängige Feldausbreitung viel einfacher abzulesen. Auch wenn ich das MATLAB Skript für die Felddarstellung im 3D-Richtdiagramm nicht abschliessen konnte, bin ich dennoch der Meinung, dass der Weg über ein MATLAB Skript für die Darstellung der Messdaten aus dem StarLab für zukünftige Arbeiten weiter zu verfolgen ist. \\  
Ich bin überzeugt, dass ich mit der Wahl dieser Bachelor Diplom Arbeit ein sehr interessantes Projekt ausgesucht habe, welches bei der Lösungsfindung viele Freiheitsgrade zuliess und für mich daher sehr lehrreich war. Die Besprechungen mit meinem Betreuer Prof. Marcel Joss waren für mich stets hilfreich, da sie mir Wege aufgezeigt haben, um Probleme und Schwierigkeiten zu umgehen oder zu meistern. Für meine weitere Laufbahn nehme ich vor allem die Erfahrungen aus dem Umgang mit dem Simulationstool mit, welche mir die Wichtigkeit eines zielgerichteten Vorgehens aufgezeigt haben.\\
Ich bin überzeugt, dass die vorliegende Arbeit der Firma Flytec AG eine Hilfe sein wird, die zukünftige Generation der Bluetooth Antenne im „Connect 1“ Gerät neu zu gestalten und somit den Funktionsumfang dieser Geräteserie zu erweitern.
