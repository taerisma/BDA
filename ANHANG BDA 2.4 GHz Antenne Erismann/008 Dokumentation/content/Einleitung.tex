\newpage
\chapter{Einleitung}

Gleitschirmpiloten müssen sich während eines Fluges stets orientieren können, um wieder sicher auf dem Boden zu landen. Da die Orientierung in der Luft ohne fixe Orientierungspunkte sehr anspruchsvoll ist, greifen die meisten Piloten auf technische Hilfsmittel zurück. Beispielsweise kommen Variometer zur Anwendung, welche den Luftdruck messen und mit einem Piepton dem Piloten anzeigen, ob er sich im Steig- oder Sinkflug befindet. Des Weiteren werden Barometer dazu verwendet die relative Höhe aufgrund der Luftdruckänderung aufzuzeichnen. Ausserdem nutzen viele Piloten GPS-Geräte, um ihre Flugrouten aufzuzeichnen. Aus diesen Daten kann ein Pilot unter anderem auf de Thermik schliessen.\\ Um den Piloten unabhängig vom Hilfsgerät zu machen, erfolgt die Informationsvermittlung, wie bereits erwähnt, teilweise akustisch. Die Firma Flytec stellt seit über 30 Jahren verschiedene Fluginstrumente für die Tuchfliegerei her. Ihre Instrumente können am Rumpf oder am Oberschenkel getragen werden und die Informationen werden auf einer berührungssensitiven Anzeige dem Piloten zur Verfügung gestellt.\\


In dieser Arbeit soll eine Kompaktantenne entwickelt werden, welche im Rahmen des \glqq near pilot network\grqq \ zur Anwendung kommen wird. Dabei sollen in Zukunft die Fluggeräte über eine Bluetooth-Verbindung mit einem Smartphone gekoppelt werden. In den folgenden Abschnitten wird die Ausgangslage dokumentiert und das bisherige Antennensystem beschrieben. In einem weiteren Schritt wird die Theorie der Kompaktantennen erarbeitet, um das Abstrahlverhalten der verschiedenen Antennen besser zu verstehen und die anschliessenden Simulationen sowie die Antennenmessungen fundiert interpretieren zu können. Aus einer Vorauswahl wird das vielversprechendste Konzept ausgewählt und für den Einsatz in einem Handgerät optimiert. Abschliessend soll ein Fazit gezogen und weitere Entwicklungsmöglichkeiten vorgeschlagen werden.

