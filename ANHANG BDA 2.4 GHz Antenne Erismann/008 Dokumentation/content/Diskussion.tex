\chapter{Diskussion}
 Im vorliegenden Kapitel werden  die erhaltenen Resultate mit den Erwartungen aus der Theorie und den Simulationen, welche in den Kapitel \ref{sec:Theorie}
 und \ref{sec:Sim_verschiedene_Geometrien}
 aufgeführt sind, verglichen werden.\\
Im Kapitel \ref{sec:Interpretation_Dipol} wurde festgelegt, dass als Funktionsmuster eine Dipol Antenne mit der Länge L = 50.25 mm und einer Breite von 3 mm gefertigt wird. Die Abbildung \ref{S11_Messung_Simulation_Dipolantenn_Freiraum} im Kapitel \ref{sec:Messergebinsse} zeigt die simulierten $S_{11}$-Verläufe der 3 mm breiten und 50.25 mm langen Dipol Antenne im Freiraum und im Gerät als rote und blaue Kurve. Hierbei zeigt sich eine Verschiebung der Resonanzfrequenz $f_{res}$ zu tieferen Frequenzen bei einer Dipol Antenne im Gerät gegenüber einem Dipol im Freiraum. Die simulierte Resonanzfrequenz im Gerät ist mit 2.44 GHz und identisch zur am Funktionsmuster gemessenen Resonanzfrequenz. Das in der Simulationssoftware verwendete Modell des "Connect 1" \ Fluggerätes eignet sich somit gut, um die Verschiebung der Resonanzfrequenz durch die materialspezifischen Eigenschaften vorauszusagen.\\
Ein ganz anderes Bild zeigt sich bei der Rückflussdämpfung $S_{11}$. Hierbei fallen die gemessenen |$S_{11}$|-Werte des Funktionsmusters viel grösser aus als durch die Simulation einer Dipol Antenne im Gerät zu erwarten gewesen war. Bemerkenswert ist dies insbesondere, da die Simulation der Dipol Antenne im Gerät eine erheblich kleinere Rückflussdämpfung ergab als die Simulation des Dipols im Freiraum, die gemessene Rückflussdämpfung des Funktionsmusters jedoch eher den Werten der simulierten Dipol Antenne im Freiraum entspricht. Eine -10 dB-Bandbreite ist bei der simulierten Antenne im Gerät nicht auszumachen, da der $S_{11}$-Wert lediglich -5.25 dB bei der Resonanzfrequenz beträgt, während die gemessene -10 dB-Bandbreite 190 MHz und die -10 dB-Bandbreite der simulierten Antenne im Freiraum 360 MHz beträgt. Das Simulationsmodell für das "Connect 1" \  Fluggerät eignet sich somit nicht für Voraussage der Rückflussdämpfung in Abhängigkeit der Antennenumgebung. Es von einem Modellfehler auszugehen, welcher durch einen weiteren Ausbau des Simulationsmodells vermutlich behoben werden kann.\\
Das Funktionsmuster weist betragsmässig grösseren $S_{11}$-Werte auf, als durch die Simulation zu erwarten war. Eine Erklärung für die überraschend guten $S_{11}$-Werte des Funktionsmusters findet sich wahrscheinlich bei der Antennenimpedanz $Z_{ant}$. Diese zeigte in der Messung des Funktionsmusters einen Wert von (30+j4) $\Omega$ bei der Zielfrequenz von 2.45 GHz, weshalb eine gute Anpassung an die Quellenimpedanz des Netzwerkanalysators von (50+j0) $\Omega$ zustande kommt. Die Simulation der Dipol Antenne im Gerät zeigte eine Antennenimpedanz $Z_{ant}$ von (16+j14.7) $\Omega$  mit der entsprechenden Abstrahleffizienz von $\eta$ = 65 $\%$. Die Abstrahleffizienz $\eta$ des Funktionsmusters konnte mit dem StarLab gemessen werden. In Abbildung \ref{S11_Messung_Simulation_Dipolantenn_Freiraum} im Kapitel \ref{sec:Messergebinsse} ist die gemessene Abstrahleffizienz über einen Frequenzbereich von 2.4 bis 2.5 GHz dargestellt. Bei der Zielfrequenz von 2.45 GHZ ist eine Effizienz von 49 $\%$ abzulesen, was nicht ganz dem aufgrund der Simulation erwarteten Wert entspricht. Dennoch zeigte sich, dass durch die Wahl eines geeigneten Antennendesigns  eine wesentlich höhere Abstrahleffizienz erreicht wird als mit der bis anhin verwendeten Monopol Antenne und dies ohne den aufwändigen Einsatz eines Anpassnetzwerkes.\\
Um die Abstrahleffizienz  zu erhöhen, sollte das Antennendesign so optimiert werden, dass die Antennenimpedanz $Z_{ant}$ der komplexen Impedanz des Transceivers (70+j30) $\Omega$ entspricht. Um Wellenanpassung zu erreichen, muss demnach eine Antenne mit 70 $\Omega$ Realanteil und einem induktive Imaginäranteil von 30 $\Omega$ gefunden werden. Eine weitere Möglichkeit der Impedanzanpassung bildet die Phasentransformation mit Hilfe einer längshomogenen Leitung. Die Analyse der Wellenanpassung an den Transceiver konnte in dieser Arbeit aus zeitlichen nicht gemacht werden, die theoretischen Grundlagen sind jedoch bereits beschrieben. \\
Ein weiterer interessanter Vergleich zwischen den Messresultaten und der Theorie ergibt sich bezüglich der Abstrahlcharakteristik der Dipol Antenne. Aus der Theorie erwartet man eine torusförmige Abstrahlung der elektromagnetischen Wellen radial um die Dipol Antenne, der Abbildung \ref{fig:DipolEFerd} entsprechend. Die Simulationen der Dipol Antenne sowie die Messungen des Funktionsmusters zeigen jedoch eine nicht kreisförmige Feldausbreitung, wie aus Abbildung \ref{fig:Schnittgemessen} im Kapitel {sec:Messergebinsse} ersichtlich ist. Während erstere Abbildung je eine Nullstelle des elektromagnetischen Feldes in der Verlängerung der Dipolachse zeigt, sind aus letzterer Abbildung keine eindeutigen Nullstellen ersichtlich. Jedoch zeigen sich entlang der Dipol-Ausrichtung Einbuchtungen der Feldstärke, welche den theoretischen Nullstellen zu entsprechen scheint. Die gemessene Abstrahlcharakteristik des Funktionsmusters weist zudem auch entlang der x-Achse eine Signaldämpfung auf. Eine Erklärung hierfür könnte in den absorbierenden Eigenschaften der Antennenumgebung mit den damit verbundenen dielektrischen Verlusten gefunden werden, da das Fluggerät im kartesischen Koordinatensystem auf der positiven x-Achse zu liegen kommt. Als weitere Erklärung für die schwache Feldabstrahlung im Ursprung  kann zudem angeführt werden, dass bei Betreibung eines Dipols in Resonanz an der Speisestelle ein Kontenpunkt entsteht und somit im Zentrum des $\lambda/2$-Dipols die kleinste Spannung herrscht. \\
Aus den Simulationsdateien bezüglich der Abstrahlcharakteristik der Dipol Antenne  im Gerät und dem qualitativen Abstrahldiagramm des Funktionsmusters in Abbildung \ref{fig:3D Richtdiagramm} kann entnommen werden, dass mit einer Kompaktantenne in einem mobilen Gerät kaum ein isotropes Abstrahlverhalten zu erreichen ist. Jedoch ist ein quasi kugelförmiges elektromagnetisches Feld um die Antenne möglich. Dieses impliziert Schwankungen der Feldstärke, wodurch mit Verbindungsunterbrüchen zwischen Sender und Empfänger zu rechnen ist. Um diese Unterbrüche zu reduzieren bleiben zwei Ansatzpunkte. Zum einem kann durch Erhöhung der Antennenabstrahleffizienz der Senderadius der Antenne und somit die Feldstärke gesteigert werden, dabei bleibt die Abstrahlcharakteristik jedoch identisch. Andererseits kann durch Optimierung des Antennendesigns und der Positionierung der Antenne versucht werden ein isotropes Abstrahlverhalten zu erreichen. \\
Das in dieser Arbeit gefundene Funktionsmuster entspricht grösstenteils den Designanforderungen. Das vorgegebene Antennenvolumen wird mit dem gewählten Antennendesign eingehalten, ebenso wird die gewünschte Sendebandbreite von 2.4 GHz bis 2.5 GHz durch die -10 dB-Bandbreit abgedeckt. Zudem hat neu etablierte symmetrische Antennendesign den Vorteil, dass auf den bis anhin verwendeten Balun verzichtet  werden kann. Weiter ist der gezeigte Wirkungsgrad mit rund 50 $\%$ im Bluetooth Sendespektrum sehr viel grösser als der Wirkungsgrad der bisher verwendeten Bluetooth Antenne.\\
Trotz der zahlreich erfüllten Designanforderungen empfiehlt sich das Antennendesign weiter zu optimieren, um eine noch höhere Abstrahleffizienz und damit verbunden einen grösseren Wirkungsradius der Antenne zu erreichen. Dies kann meiner Meinung nach primär durch eine verbesserte Anpassung an den Tranceiver erreicht werden. 

 




 
